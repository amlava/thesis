\chapter{Sequential Proximal SGD Method}

In this chapter we will analyze the convergence
properties of \cref{alg:seqprox-sgd} applied to
the constrained stochastic optimization problem
\begin{equation}
    \label{eq:main-problem-copy}
    \tag{P}
    \begin{aligned}
    &\min_{x \in \mathbb{R}^n} \, \{\, f(x) := \E(F_\xi(x)) + r(x) \,\} \\
    &\textup{s.\,t.} \quad A(\xi) x - b(\xi) \leq 0 \quad \textnormal{a.\,s.},
    \end{aligned}
\end{equation}
where we implicitly assume the existence of a probability space
$(\Omega, \mathcal{F}, \Prob)$ on which the random variable $\xi\colon\Omega\to\R^p$,
as well as the expected value mapping $\E(\cdot)$,
are defined. Further, we endow the probability space with a filtration $\mathcal{F} := (\mathcal{F}_k)_{k\in\N}$
defined by
\[
    \mathcal{F}_k := \sigma(\xi_0,\dots,\xi_{k-1})
\]
and denote the conditional expectation given $\mathcal{F}_k$ as
\[
    \E_k(X) := \E(X \mid \mathcal{F}_k)
\]
for all $k\in\N$. Similarly, we write
\[
    \Var_k(X) := \E_k\norm{X - \E_k(X)}^2
\]
for the conditional variance given $\mathcal{F}_k$, for all $k\in\N$.
Note that the sequence of iterates $(x_k)_{k\in\N}$ generated by \cref{alg:seqprox-sgd}
is adapted to $\mathcal{F}$ and thus $\E_k(x_k) = x_k$ for all $k\in\N$.
Finally, we denote the feasible set for problem \eqref{eq:main-problem-copy} as
\[
    \mathcal{X} := \{\, x\in\dom(f) \mid A(\xi)x - b(\xi) \leq 0 \enspace \textnormal{a.\,s.} \,\}.
\]
\begin{assumption}
    \label{ass:basic-assumptions}
    Problem \eqref{eq:main-problem} satisfies the following:
    \begin{enumerate}
        \item The function $x\mapsto F_\xi(x)$ is almost surely $L$-smooth for some $L\in(0,\infty)$,
            and there exists a point $x\in\R^n$ such that $\E\norm{\nabla F_\xi(x)}^2 < \infty$.
        Further, the expectation $x\mapsto \E(F_\xi(x))$ is $\mu$-strongly convex for some
        $\mu\in(0, \infty)$.
        \item The function $r$ is proper, convex, and locally Lipschitz continuous on $\dom(r)$.
        \item The matrix-valued map $A\colon\R^p\to\R^{m\times n}$, and the vector valued
            map $b\colon\R^p\to\R^m$, are both (Borel-)measurable.
        \item The sequence $(\gamma_k)_{k\in\N_0}$ is nondecreasing and unbounded. 
        \item There exists at least one feasible point.
    \end{enumerate}
\end{assumption}
\Cref{alg:seqprox-sgd} works with penalty functions $(\pi_k)_{k\in\N_0}$.
These will always take the form
\[
    \pi_k(x) := \E(h_k(x; A(\xi), b(\xi))),
\]
where, for all $k\in\N_0$,
we let $(h_k)_{k\in\N_0}$ be any sequence of functions from
$\R^n \times \R^{m\times n} \times \R^m$ to $\R$ with the following properties:
\begin{enumerate}
    \item For all $k\in\N_0$, $h_k$ is convex and differentiable.
    \item $\pi_k(x) \geq \E\norm{(A(\xi)x - b(\xi))_+}_1$ for all $x\in\R^n$ and $k\in\N_0$.
    \item There exists a sequence $(\alpha_k)_{k\in\N_0}$ with $\lim_{k\to\infty} \alpha_k = 0$ such that
    $\pi_k(\tilde{x}) \leq \alpha_k$ for all $k\in\N_0$ and all feasible points $\tilde{x}\in\mathcal{X}$.
    \item The gradients of $(h_k)_{k\in\N_0}$ have uniformly bounded second moment:
        \[
            \sup_{k\in\N_0}\sup_{x\in\R^n} \E\norm{\nabla h_k(x; A(\xi), b(\xi))}^2 < \infty
        \]
        almost surely.
\end{enumerate}
Finally, for all $k\in\N_0$, we define the sequence of unconstrained optimization problems
\begin{equation}
    \label{eq:main-problem-penalized}
    \tag{$\textnormal{P}^\textnormal{k}$}
    \min_{x\in\R^n}
    \left\{\, f_k(x) := f(x) + \gamma_k \pi_k(x) \,\right\},
\end{equation}
where $\gamma_k\in(0,\infty)$.
Since proximal methods seperate the smooth part of the above objective,
given by $x\mapsto \E(F_\xi(x)) + \gamma_k \pi_k(x)$,
from the nonsmooth part, $x\mapsto r(x)$, it is useful to also define the functions
\[
    \psi_k(x) := \E(F_\xi(x)) + \gamma_k \pi_k(x),
\]
for all $k\in\N_0$.

\Cref{ass:basic-assumptions} has multiple useful implications, which are captured
by the following lemma.
\begin{lemma}
    \label{lem:basic-properties}
    \Cref{ass:basic-assumptions} implies the following:
    \begin{enumerate}
        \item The objective $f$ of \eqref{eq:main-problem-copy} is $\mu$-strongly convex, locally
            Lipschitz continuous, and subdifferentiable.
        \item The objectives $(f_k)_{k\in\N_0}$ of \eqref{eq:main-problem-penalized} are $\mu$-strongly convex, locally
            Lipschitz continuous, and subdifferentiable.
        \item The functions $(\psi_k)_{k\in\N_0}$ are $\mu$-strongly convex.
        \item Let $B\subset\R^n$ be a bounded subset. Then, for all $x\in B,
        \gamma\in[0,\infty), k\in\N_0$, the stochastic gradient
        \[
            g(x) := \nabla F_\xi(x) + \gamma \nabla h_k(x; A(\xi), b(\xi))
        \]
        satisfies $\sup_{x\in B} g(x) = \mathcal{O}(\gamma)$ almost surely.
        \item The gradients of $(\pi_k)_{k\in\N_0}$ are uniformly bounded: There exists $G\in(0,\infty)$ such that
            \[
                \sup_{k\in\N_0} \sup_{x\in\R^n} \norm{\nabla \pi_k(x)} \leq G.
            \]
        \item There exists a unique solution $x^\star\in\mathcal{X}$ for \eqref{eq:main-problem-copy} and, for all $k\in\N$,
            there exists a unique solution $x_k^\star\in\R^n$ for \eqref{eq:main-problem-penalized}.
    \end{enumerate}
\end{lemma}
\begin{proof}
    
\end{proof}
The fact that $x_k^\star$ must not be
feasible introduces difficulties that prevent the use of
standard arguments from the SGD literature to analyse convergence. Our proof
methods combine approaches from recent works, mainly
the already mentioned \cite{nedich2023huber}, as well as
\cite{cutler2023stochastic}. The latter paper
investiages \textit{stochastic optimization problems under distributional drift}.
While these kinds of problems are not exactly the same as the ones we are
working with, the two settings do indeed exhibit striking resemblences.
Namely, Cutler et al. \cite{cutler2023stochastic} investigated a sequence of
time-dependent composite problems of the form
\[
    \min_{x\in\R^n} g_t(x) + r_t(x),
\]
where, for all $t\in\N$, $g_t$ is smooth strongly convex, and $r_t$
is convex. Comparing to our problem \eqref{eq:main-problem-penalized},
rewritten as 
\begin{equation}
    \label{eq:main-problem-penalized-copy-with-psi}
    \tag{$\textnormal{P}^\textnormal{k}$}
    \min_{x\in\R^n}
    \left\{\, f_k(x) = \psi_k(x) + r(x) \,\right\},
\end{equation}
we see that the two settings almost match, except for smoothness properties. Namely,
we do not need the family $(\psi_k)_{k\in\N}$ itself to be smooth, but only that
it is composed of an $L$-smooth function and a differentiable one
with uniformly bounded gradients. Even though it would be realistic to
assume Lipschitz smoothness of the penalties $\pi_k$, this would
lead to smoothness constants that blow up for $k\to\infty$, in order to
satisfy the convergence assumption $\lim_{k\to\infty} \pi_k = \pi_{\ell_1}$.
A naive application of the techniques in Cutler et al. \cite{cutler2023stochastic}
would then lead to a restriction on the step sizes $(\eta_k)_{k\in\N}$ of
the form $\eta_k \in (0, 1/L_k)$ with $L_k$ the smoothness constant of $\psi_k$.
With some care however, we manage to circumvent this restriction, as well as
the Lipschitz assumption on $\nabla \pi_k$, and only require $\eta_k\in(0, 1/L)$,
which allows for a lot more freedom in the choice of step sizes.
The resulting inequality can then be used
to adapt the proof strategy from Nedić et al. \cite{nedich2023huber}
(where the authors did not use proximal maps in their algorithm)
to our proximal method.

\section{Almost sure convergence}
\label{sec:almost-sure-convergence}

In this section we will establish
conditions under which we can guarantee almost sure convergence of the sequence of iterates $(x_k)_{k\in\N_0}$.
The proof will also yield convergence in expectation of
$(x_k)_{k\in\N_0}$ to $x^\star$ along a subsequence. The main use of the almost sure convergence
result will be that we will have conditions on the stepsizes $(\eta_k)_{k\in\N_0}$ and the
penalty parameters $(\gamma_k)_{k\in\N_0}$ to ensure that $(x_k)_{k\in\N_0}$ is bounded with
probability one. This, together with some of the results we prove along the way, will
come in very handy in the subsequent analysis of
the quantitative convergence rates of our methods.

The proof for almost sure convergence hinges on two technical lemmata. The first is the well-known
Robbins-Siegmund lemma, which provides a general sufficient condition to guarantee almost sure
convergence of so-called "almost supermartingales".
\begin{lemma}[Robbins-Siegmund]
    \label{lem:robbins-siegmund}
    Let $(\mathcal{F}_k)_{k\in\N}$ be an increasing sequence of $\sigma$-algebras and $v_k, a_k,
    b_k, c_k$ be nonnegative $\mathcal{F}_k$-measurable random variables. If, for all $k\in\N$,
    \begin{equation}
        \label{eq:lem:robbins-siegmund-almost-supermartingales}
        \E(v_{k+1} \mid \mathcal{F}_k) \leq v_k(1+a_k) + b_k - c_k,        
    \end{equation}
    and $\sum_{k=1}^\infty a_k < \infty, \sum_{k=1}^\infty b_k < \infty$ a.\,s., then with probability
    one, $(v_k)_{k\in\N}$ is convergent and it holds that $\sum_{k=1}^\infty c_k < \infty$.
\end{lemma}
\begin{proof}
    See \cite{robbins1971convergence}.
\end{proof}

Our goal for the rest of this section is to derive a recursive inequality for the
sequence $\norm{x_k - x^\star}^2$, which resembles \eqref{eq:lem:robbins-siegmund-almost-supermartingales}.
In order to do this, we will first analyze the behavior of the
sequence of solutions $(x_k^\star)_{k\in\N_0}$ of the sequence of unconstrained
problems \eqref{eq:main-problem-penalized}. We can then use this to "build a bridge"
between $(x_k)_{k\in\N_0}$ and $x^\star$, by considering their respective relationships to
the sequence $(x_k^\star)_{k\in\N_0}$.

The first step towards analyzing the convergence of $(x_k^\star)_{k\in\N_0}$
will be to (locally) bound the distance $\dist(x,\mathcal{X})$
by (a term proportional to) the penalty $\pi_k(x)$.
We will rely on
an extension of a classic result by Hoffman \cite{hoffman2003approximate},
who analyzed the distance of points $x\in\R^n$ to the set of solutions of linear systems of inequalities
$Ax \leq b$ with $A\in\R^{m\times n}$, $b\in\R^m$.
Crucially, there always exists a constant $\tau\in(0,\infty)$, such that
\[
    \tau \dist(x, S) \leq \norm{(Ax - b)_+}_\infty,
\]
where $S := \{\, y\in\R^n \mid Ay \leq b \,\}$.
Since we are essentially dealing with infnitely long matrices,
we cannot directly apply Hoffman's lemma. Instead,
we will use the theory of metric subregularity.
Before we proceed, we make two more assumptions.
\begin{assumption}
    \label{ass:compact-support-without-zero}
    There exists a compact set $\Xi\in\R^p$ such that $\xi$ is supported
    on $\Xi$ and $\Prob^\xi$ admits a Lebesgue-density $q$ such that
    \begin{enumerate}
        \item $q$ is continuous on $\Xi$ and
        \item $q(x) > 0$ for all $x\in \Xi$.
    \end{enumerate}
    Further, we assume that $\int_{\Xi} \norm{A(z)x - b(z)}_1 \, dz < \infty$ for all $x\in\R^n$.
\end{assumption}
A simple sufficient condition for the last part of the assumption is continuity of the maps $A$ and $b$
on $\Xi$.

\begin{assumption}
    \label{ass:slater-point}
    There exists a \textit{Slater point}, i.\,e. a point $x\in\mathcal{X}$ such that $A(\xi)x < b(\xi)$
    almost surely.
\end{assumption}
% Notably, it is here where we make use of 
% our assumption that $(\pi_k)_{k\in\N}$ upperbounds
% the hinge-penalty,
% \[
%     \pi_k(x) \geq \E\norm{(A(\xi)x - b(\xi))_+}_{1},
% \]
% for all $x\in\R^n$ and $k\in\N$.
We will now formulate and prove the lemma.
\begin{lemma}[Subfeasibility bound via penalty functions]
    \label{lem:lower-bound-on-penalties-by-distance-to-feasibility}
    Assume \cref{ass:basic-assumptions,ass:compact-support-without-zero,ass:slater-point} hold
    and let $C\subset\R^n$ be a compact subset. Then
    there exists a constant $\tau\in(0,\infty)$ such that
    \[
       \tau \dist(x, \mathcal{X}) \leq \pi_k(x),
    \]
    for all $x\in C$ and $k\in\N_0$.
\end{lemma}
\begin{proof}
    We consider the Banach space $Y := L_1(\Xi, \R^m)$ equipped with the norm
    \[
        \norm{y}_{Y} := \int_{\Xi} \norm{y(z)}_1 \, dz.
    \]
    The distance map on $Y$, denoted $\dist_Y$, is then given by
    \[
        \dist_Y(y, S) := \inf_{y^\prime\in S} \norm{y - y^\prime}_Y
    \]
    for $y\in Y$ and $S\subset Y$.
    We define the multifunction $\varPsi\colon \R^n \to 2^{Y}$ as
    \[
        \varPsi(x) := \{\, y\in Y \mid A(\xi)x - b(\xi) \leq y(\xi) \text{ \,a.\,s.} \,\}.
    \]
    This multifunction is closed and convex. Indeed, let $(x_k)_{k\in\N} \in \R^n$
    be a sequence that converges to some $x\in\R^n$, and assume that there exists
    a sequence $(y_k)_{k\in\N}$ with $y_k\in\varPsi(x_k)$ for all $k\in\N$, such that
    $\lim_{k\to\infty} y_k = y \in Y$.
    Then, with probability one,
    \[
        A(\xi)x - b(\xi) = \lim_{k\to\infty} A(\xi)x_k - b(\xi) \leq \lim_{k\to\infty} y_k(\xi) = y(\xi).
    \]
    Hence $y\in\varPsi(x)$, which proves closedness.
    For convexity, let
    $x_1, x_2 \in \R^n$ and $t\in [0, 1]$. Then
    \begin{align*}
        t\varPsi(x_1) + (1-t)\varPsi(x_2)
        = \{\, y &\mid y = ty_1 + (1-t)y_2 \text{ for } y_1, y_2\in Y \\
                &\phantom{===} \text{ such that } A(\xi)x_1 - b(\xi) \leq y_1(\xi) \text{ \,a.\,s.} \\
                &\phantom{===} \text{ and } A(\xi)x_2 - b(\xi) \leq y_2(\xi) \text{ \,a.\,s.} \,\}.
    \end{align*}
    Let $x := t x_1 + (1-t) x_2$. Then, for any $y \in t\varPsi(x_1) + (1-t)\varPsi(x_2)$, there exist
    $y_1, y_2\in Y$, such that
    \[
        A(\xi)x - b(\xi) = t(A(\xi)x_1 - b(\xi)) + (1-t)(A(\xi)x_2 - b(\xi)) \leq t y_1(\xi) + (1-t)y_2(\xi) = y(\xi).
    \]
    Hence,
    \[
        t\varPsi(x_1) + (1-t)\varPsi(x_2) \subset \varPsi(t x_1 + (1-t) x_2),
    \]
    proving convexity.
    Now let $x_0\in C$ be some arbitrary point.
    Per definition of the inverse $\varPsi^{-1}$, it holds that
    \[
        \varPsi^{-1}(0) = \{\, x\in\R^n \mid 0 \in \varPsi(x) \,\} = \{\, x\in\R^n \mid A(\xi)x - b(\xi) \leq 0 \text{ \,a.\,s.} \,\}
        =
        \mathcal{X}.
    \]
    Hence we can write $\dist(x, \mathcal{X}) = \dist(x, \varPsi^{-1}(0))$.
    Note that \cref{ass:slater-point} implies $0 \in \interior(\range \varPsi )$. Therefore we
    can apply \cref{prop:robinson-ursescu}, which guarantees the existence of a constant
    $c\in\R_{\geq 0}$, such that
    \[
        \dist(x, \varPsi^{-1}(y)) \leq c \dist_Y(y, \varPsi(x))
    \]
    for all $(x, y)$ in some open neighborhood $U \subset \R^n\times Y$ containing $(x_0, 0)$.
    In particular,
    \begin{equation}
        \label{eq:proof:lem:lower-bound-on-penalties-by-distance-to-feasibility-1}
        \dist(x, \mathcal{X}) \leq c \dist_Y(0, \varPsi(x))        
    \end{equation}
    for all $x$ in the open set $U \cap (\R^n\times \{0\})$. Since $x_0$ is arbitrary,
    we can derive a similar bound that holds around an open neighborhood $U_x \subset \R^n$ of a point $x\in C$, for any
    $x\in C$, yielding corresponding constants $(c_x)_{x\in C}$.
    By compactness of $C$, the open covering
    \[
        C \subset \bigcup_{x\in C} U_x
    \]
    has a finite subcovering
    \[
        C \subset \bigcup_{i=1}^\ell U_{x_i}
    \]
    with $(x_i)_{i\in\{\, 1,\dots, \ell \,\}} \subset C$. The corresponding constants
    $(c_{x_i})_{i\in\{\, 1,\dots, \ell \,\}}$ have a maximum $c := \max_{i\in\{\, 1,\dots, \ell \,\}} c_{x_i}$.
    Thus, we have shown that there exists $c\in\R_{\geq 0}$ such that
    \[
        \dist(x, \mathcal{X}) \leq c \dist_Y(0, \varPsi(x))
    \]
    for all $x\in C$. Next, we will show that
    \begin{equation}
        \label{eq:proof:lem:lower-bound-on-penalties-by-distance-to-feasibility-2}
        \dist_Y(0, \varPsi(x))
        = \int_{\Xi} \norm{(A(z)x - b(z))_+}_1 \, dz = \norm{(Ax - b)_+}_Y
    \end{equation}
    for any $x\in C$.
    Fix $x\in C$ and let $y\in \varPsi(x)$.
    By definition of $\varPsi(x)$ and positivity of $q$ on $\Xi$,
    it holds that $y\in \varPsi(x) \iff A(z)x - b(z) \leq y(z)$ for all $z\in \Xi$.
    Set
    \[
        \phi(z) := (A(z)x - b(z))_+
    \]
    for $z\in \Xi$. Clearly, $\phi \in \varPsi(x)$.
    We will show that $\norm{\phi(z)} \leq \norm{y(z)}$
    for all $z\in \Xi$. We denote by $\phi_i(z), a_i(z), b_i(z), y_i$, the $i$th
    row of $\phi(z), A(z), b(z), y$. We have
    \[
        |\phi_i(z)| = \begin{cases}
            0, &a_i (z)x - b_i(z) \leq 0 \\
            a_i (z)x - b_i(z), &\text{else}
        \end{cases}
    \]
    and thus
    \[
        |y_i| \geq \begin{cases}
            a_i(z)x - b_i(z), &\text{if } a_i(z)x - b_i(z) \geq 0, \\
            0, &\text{else.}
        \end{cases}
        \enspace =
        |\phi_i(z)|,
    \]
    for all $i\in\{\,1,\dots,m\,\}$.
    It follows that
    \begin{align*}
        \norm{\phi(z)}_1 &= \sum_{i=1}^m |\phi_i(z)| \\
                      &\leq \sum_{i=1}^m |y_i| \\
                      &= \norm{y}_1,
    \end{align*}
    and therefore,
    \[
        \inf_{y \in \varPsi(x)} \int_{\Xi} \norm{y}_1 \, dz
        \geq
        \int_{\Xi} \norm{\phi(z)} \, dz,
    \]
    proving \eqref{eq:proof:lem:lower-bound-on-penalties-by-distance-to-feasibility-2}.
    To finish the proof, we need to establish a relationship between
    \eqref{eq:proof:lem:lower-bound-on-penalties-by-distance-to-feasibility-2}
    and $\pi_k(x)$.
    By compactness of $\Xi$ and continuity of $q$, the image $q(\Xi)$
    must be compact.
    In particular, since $q(x) > 0$ for all $x\in \Xi$, there must
    exist some uniform positive lower bound $c_q \in (0, \infty)$ such that $q(x) \geq c_q$
    for all $x\in \Xi$. If we denote the Lebesgue-measure by $\lambda$, we see that
    \[
        \Prob^\xi(A) = \int_{A} q(z) \, dz \geq c_q \lambda (A),
    \]
    for all measurable $A \subset \Xi$.
    Therefore, $\Prob^\xi$ and $\lambda$ are equivalent on $\Xi$, and
    $q^{-1}$ is a $\Prob^\xi$-density of $\lambda$.
    We thus have
    \begin{align*}
        \norm{y}_Y &= \int_{z\in \Xi} \norm{y(z)}_1 \, dz \\
        &= \int_{z\in \Xi} \norm{y(z)}_1 q^{-1}(z) \, \Prob^\xi(dz) \\
        &\leq c_q^{-1} \int_{z\in \Xi} \norm{y(z)}_1 \, \Prob^\xi(dz) \\
        &= c_q^{-1} \int \norm{y(\xi)}_1 \, d\Prob \\
        &= c_q^{-1} \, \E\norm{y(\xi)}_1.
    \end{align*}
    Combining with \eqref{eq:proof:lem:lower-bound-on-penalties-by-distance-to-feasibility-1}
    and \eqref{eq:proof:lem:lower-bound-on-penalties-by-distance-to-feasibility-2}, we
    obtain
    \[
        \dist(x, \mathcal{X}) \leq c c_q^{-1} \E\norm{(A(\xi)x - b(\xi))_+}_1.
    \]
    The claim follows after setting $\tau := c^{-1} c_q$
    and applying one of the defining properties of $(\pi_k)_{k\in\N_0}$.
\end{proof}
\begin{theorem}[Convergence of $x_k^\star$]
    \label{thm:distance-to-feasibility-bound}
    Assume \cref{ass:basic-assumptions,ass:slater-point,ass:compact-support-without-zero} hold.
    Then, for all $k\in\N_0$, it holds that
    \[
        \frac{\mu}{2} |\!|x^\star - x_k^\star|\!|^2 + \frac{\mu}{2} |\!|x^\star - \proj(x_k^\star)|\!|^2
        + (\tau\gamma_k - M) \, \dist(x_k^\star, \mathcal{X})
        \leq
        \gamma_k \alpha_k,
    \]
    where $M,\tau\in (0, \infty)$ are constants.
    In particular,
    \[
        |\!|x^\star - x_k^\star|\!|^2 = \mathcal{O}(\gamma_k \alpha_k) \quad \textnormal{and} \quad
        \dist(x_k^\star, \mathcal{X}) = \mathcal{O}(\alpha_k).
    \]
\end{theorem}
\begin{proof}
    Let $k\in\N_0$. By optimality of $x_k^\star$ for $f_k$, there exists
    $0 \in \partial f_k(x_k^\star)$. Hence, by
    strong convexity, we obtain
    \begin{equation}
        \label{proof:thm:surrogate-bound-for-huber-penalty:eq1}
        \frac{\mu}{2} |\!|x^\star - x_k^\star|\!|^2
        \leq f_k(x^\star) - f_k(x_k^\star)
        = f(x^\star) - f(x_k^\star) + \gamma_k\pi_k(x^\star) - \gamma_k\pi_k(x_k^\star).
    \end{equation}
    For the rest of this proof, all subgradients of $f$ will refer to the
    minimum-norm subgradient.
    We can write
    \begin{align*}
        f(x^\star) - f(x_k^\star) &= f(x^\star) - f(\proj(x_k^\star)) + f(\proj(x_k^\star)) - f(x_k^\star) \\
                                  &\leq -\frac{\mu}{2} |\!|x^\star - \proj(x_k^\star)|\!|^2
                                    - \langle \tilde{\nabla} f(x^\star), \proj(x_k^\star) - x^\star \rangle + f(\proj(x_k^\star)) - f(x_k^\star), 
    \end{align*}
    where we again used strong convexity in the second step.
    Since $\proj(x_k^\star)\in\mathcal{X}$ and $x^\star$ is optimal for $f$ on $\mathcal{X}$,
    it holds that
    \[
        \langle \tilde{\nabla} f(x^\star), \proj(x_k^\star) - x^\star \rangle \geq 0,
    \]
    and thus
    \[
        f(x^\star) - f(x_k^\star) \leq -\frac{\mu}{2} |\!|x^\star - \proj(x_k^\star)|\!|^2 + f(\proj(x_k^\star)) - f(x_k^\star).
    \]
    To continue with the proof, we will need to first show that $(x_k^\star)_{k\in\N}$
    is bounded: By strong convexity and optimality of $x_k^\star$ for $f_k$, for any feasible $x\in\mathcal{X}$, it holds that
    \begin{align*}
        \frac{\mu}{2} \norm{x - x_k^\star}^2
                &\leq f_k(x) - f_k(x_k^\star) \\
                &= f(x) + \gamma_k\pi_k(x) - f_k(x^\star_k) \\
                &\leq f(x) + \gamma_k \alpha_k - f_k(x^\star_k) \\
                &\leq f(x) + \gamma_k \alpha_k - f_\text{min}\,,
    \end{align*}
    where $f_\text{min}$ is the minimum value of $f$, which exists by strong convexity.
    Noting that $\lim_{k\to\infty} \gamma_k\alpha_k = 0$ per one of our assumptions, it follows
    that $(x_k^\star)_{k\in\N}$ must be a bounded sequence.
    Hence, by
    local Lipschitz continuity of $f$ (\cref{lem:basic-properties}) and Cauchy-Schwarz, we have
    \[
        f(\proj(x_k^\star)) - f(x_k^\star) \leq M \dist(x_k^\star, \mathcal{X}),
    \]
    for some constant $M\in[0,\infty)$
    We now obain
    \[
        f(x^\star) - f(x_k^\star) \leq -\frac{\mu}{2} |\!|x^\star - \proj(x_k^\star)|\!|^2 + M \dist(x_k^\star, \mathcal{X}).
    \]
    Plugging this into \eqref{proof:thm:surrogate-bound-for-huber-penalty:eq1}, we obtain
    \[
        \frac{\mu}{2} |\!|x^\star - x_k^\star|\!|^2
        \leq
        -\frac{\mu}{2} |\!|x^\star - \proj(x_k^\star)|\!|^2 + M \dist(x_k^\star, \mathcal{X})
        + \gamma_k\pi_k(x^\star) - \gamma_k\pi_k(x_k^\star).
    \]
    Now, using our lower bound on $\pi_k(x_k^\star)$ from \cref{lem:lower-bound-on-penalties-by-distance-to-feasibility},
    and combining terms, we arrive at the inequality
    \[
        \frac{\mu}{2} |\!|x^\star - x_k^\star|\!|^2
        \leq
        -\frac{\mu}{2} |\!|x^\star - \proj(x_k^\star)|\!|^2 + (M - \gamma_k\tau) \dist(x_k^\star, \mathcal{X})
        + \gamma_k\pi_k(x^\star).
    \]
    Using $\pi_k(x^\star) \leq \alpha_k$ and rearranging, we obtain
    \[
        \frac{\mu}{2} |\!|x^\star - x_k^\star|\!|^2
        +
        \frac{\mu}{2} |\!|x^\star - \proj(x_k^\star)|\!|^2
        +
        (\gamma_k\tau - M) \dist(x_k^\star, \mathcal{X})
        \leq
        \gamma_k \alpha_k.
    \]
    The asymptotic rate for $|\!|x^\star - x_k^\star|\!|^2$ now follows.
    For the bound on $\dist(x_k^\star, \mathcal{X})$, we let $K$ be large enough such that
    $\gamma_k\tau > M$ for all $k \geq K$. Dividing by $\gamma_k$ on both sides and using the nonnegativity
    of the other terms on the left-hand side, we get
    \[
        c \cdot \dist(x_k^\star, \mathcal{X})
        \leq
        \frac{\gamma_k\tau - M}{\gamma_k} \dist(x_k^\star, \mathcal{X})
        \leq
        \alpha_k,
    \]
    for all $k \geq K$ and some constant $c\in(0,1)$, as desired.
\end{proof}

Having established the convergence of the sequence $(x_k^\star)_{k\in\N_0}$
to $x^\star$, we will now shift our attention to the iterates $(x_k)_{k\in\N_0}$
of \cref{alg:seqprox-sgd}. We begin with the following
fundamental recursive inequality.
% \begin{lemma}[One-step improvement]
%     \label{lem:one-step-improvement}
%     Let $\eta_k\in(0, 1/L_k)$ for all $k\in\N$. Then
%     the iterates $(x_k)_{k\in\N}$ generated by \cref{alg:seqprox-sgd} with step size
%     schedule $(\eta_k)_{k\in\N}$ satisfy
%     \[
%         2\eta_k\E_k(f_k(x_{k+1}) - f_k(x))
%         \leq
%         (1-\mu\eta_k)\norm{x - x_k}^2
%         - \E_k\norm{x - x_{k+1}}^2
%         + \frac{\eta_k^2}{1 - L_k\eta_k}\Var_k(g_k),
%     \]
%     for all $x\in\R^n$ and $k\in\N$.
% \end{lemma}
% \begin{proof}
%     This proof essentially follows the proof of lemma 23 in Cutler et al. \cite{cutler2023stochastic} verbatim,
%     apart from some added details and the small difference that our smoothness constant is not independent of $k\in\N$.
%     For $k\in\N$ we denote by $g_k$ the stochastic gradient of $\psi_k$ at $x_k$
%     that is used in iteration $k$ of \cref{alg:seqprox-sgd}. By $L_k$ smoothness of $\psi_k$, we have
%     \begin{align*}
%         f_k(x_{k+1})
%         &= \psi_k(x_{k+1}) + r(x_{k+1}) \\
%         &\leq \psi_k(x_k) + \langle \nabla \psi_k(x_k), x_{k+1} - x_k \rangle + \frac{L_k}{2}\norm{x_{k+1} - x_k}^2 + r(x_{k+1}) \\
%         &= \psi_k(x_k) + r(x_{k+1}) + \langle g_k, x_{k+1} - x_k \rangle
%                          + \frac{L_k}{2}\norm{x_{k+1} - x_k}^2 + \langle z_k, x_{k+1} - x_k \rangle.
%     \end{align*}
%     By Cauchy-Schwarz and Young's inequality, we have
%     \[
%         \langle z_k, x_{k+1} - x_k \rangle \leq \frac{\delta_k}{2}\norm{z_k}^2 + \frac{1}{2\delta_k}\norm{x_{k+1} - x_k}^2
%     \]
%     for all $\delta_k\in(0,\infty)$. Therefore,
%     \begin{align*}
%         f_k(x_{k+1})
%         &\leq \psi_k(x_k) + r(x_{k+1}) + \langle g_k, x_{k+1} - x_k \rangle
%                          + \frac{L_k + \delta_k^{-1}}{2}\norm{x_{k+1} - x_k}^2
%                          + \frac{\delta_k}{2}\norm{z_k}^2 \\
%         &= \psi_k(x_k) + r(x_{k+1}) + \langle g_k, x_{k+1} - x_k \rangle
%                          + \frac{1}{2\eta_k}\norm{x_{k+1} - x_k}^2 \\
%                          &\phantom{===}+ \frac{L_k + \delta_k^{-1} - \eta_k^{-1}}{2}\norm{x_{k+1} - x_k}^2
%                          + \frac{\delta_k}{2}\norm{z_k}^2.
%     \end{align*}
%     From the definition of the proximal map, it follows that
%     \begin{align*}
%         x_{k+1} &= \prox_{\eta_k r} (x_k - \eta_k g_k) \\
%                 &= \argmin_{x\in\R^n} \Big\{\, r(x) + \frac{1}{2\eta_k}\norm{x - (x_k - \eta_k g_k)}^2 \,\Big\} \\
%                 &= \argmin_{x\in\R^n} \Big\{\, r(x) + \langle g_k, x - x_k \rangle + \frac{1}{2\eta_k}\norm{x - x_k}^2 \,\Big\},
%     \end{align*}
%     where the last step follows from expanding the square and dropping
%     the constant term $\eta_k^2 \norm{g_k}^2$ from the minimization.
%     The function $x\mapsto r(x) + \langle g_k, x - x_k \rangle + \frac{1}{2\eta_k}\norm{x - x_k}^2$ is $(2\eta_k)^{-1}$-strongly convex. Thus,
%     comparing with our previous bound on $f_k(x_{k+1})$,
%     we can conclude
%     \begin{align*}
%         f_k(x_{k+1}) &\leq \psi_k(x_k) + r(x) + \langle g_k, x - x_k \rangle + \frac{1}{2\eta_k}\norm{x - x_k}^2 - \frac{1}{2\eta_k}\norm{x - x_{k+1}}^2 \\
%         &\phantom{===} + \frac{L_k + \delta_k^{-1} - \eta_k^{-1}}{2}\norm{x_{k+1} - x_k}^2 + \frac{\delta_k}{2}\norm{z_k}^2,
%     \end{align*}
%     for all $x\in\R^n$. By $\mu$-strong convexity of $\psi_k$, we have
%     \begin{align*}
%         \psi_k(x_k) + r(x) + \langle g_k, x - x_k \rangle
%         &=
%         \psi_k(x_k) + r(x) + \langle \psi_k(x_k), x - x_k \rangle + \langle z_k, x_k - x \rangle \\
%         &\leq
%         \psi_k(x) - \frac{\mu}{2}\norm{x - x_k}^2 + r(x) + \langle z_k, x_k - x \rangle \\
%         &=
%         f_k(x) - \frac{\mu}{2}\norm{x - x_k}^2 + \langle z_k, x_k - x \rangle,
%     \end{align*}
%     for all $x\in\R^n$. Hence,
%     \begin{align*}
%         f_k(x_{k+1})
%         &\leq f_k(x) + \Big(\frac{1}{2\eta_k} - \frac{\mu}{2}\Big)\norm{x - x_k}^2 + \langle z_k, x_k - x \rangle 
%             - \frac{1}{2\eta_k}\norm{x - x_{k+1}}^2 \\
%         &\phantom{===} + \frac{L_k + \delta_k^{-1} - \eta_k^{-1}}{2}\norm{x_{k+1}-x_k}^2 + \frac{\delta_k}{2}\norm{z_k}^2.
%     \end{align*}
%     Choosing $\delta_k := \eta_k(1 - L_k\eta_k)^{-1}$,
%     we can drop the $\norm{x_{k+1} - x_k}^2$ term and get
%     \begin{align*}
%         f_k(x_{k+1})
%         &\leq f_k(x) + \Big(\frac{1}{2\eta_k} - \frac{\mu}{2}\Big)\norm{x - x_k}^2 + \langle z_k, x_k - x \rangle 
%             - \frac{1}{2\eta_k}\norm{x - x_{k+1}}^2 \\
%         &\phantom{===} + \frac{\eta_k}{2(1 - L_k\eta_k)}\norm{z_k}^2.
%     \end{align*}
%     Subtracting by $f_k(x)$ and multiplying both sides by $2\eta_k$
%     yields
%     \begin{align*}
%         2\eta_k(f_k(x_{k+1}) - f_k(x))
%         &\leq ( 1 - \mu\eta_k ) \norm{x - x_k}^2 + 2\eta_k\langle z_k, x_k - x \rangle 
%             - \norm{x - x_{k+1}}^2 \\
%         &\phantom{===} + \frac{\eta_k^2}{1 - L_k\eta_k}\norm{z_k}^2.
%     \end{align*}
%     By properties of conditional expectation and the definition of
%     stochastic gradients, we know that $\E_k \langle z_k, x_k - x \rangle = 0$, and
%     thus, applying conditional expectations to both sides of the above
%     inequality, we obtain
%     \[
%         2\eta_k\E_k(f_k(x_{k+1}) - f_k(x))
%         \leq ( 1 - \mu\eta_k ) \norm{x - x_k}^2
%             - \E_k\norm{x - x_{k+1}}^2 + \frac{\eta_k^2}{1 - L_k\eta_k}\E_k\norm{z_k}^2,
%     \]
%     for all $x\in\R^n$. Since $\E_k\norm{z_k}^2 = \E_k\norm{g_k - \nabla \psi_k(x_k)}^2$ and
%     $\E_k(g_k) = \nabla \psi_k(x_k)$, it holds that $\E_k\norm{z_k}^2 = \Var_k(g_k)$,
%     and we arrive at the desired result.
% \end{proof}
\begin{lemma}[One-step improvement]
    \label{lem:one-step-improvement}
    Let \cref{ass:basic-assumptions} hold and
    let $\rho\in(0,1)$ and $\eta_k\in(0, \rho L^{-1}]$ for all $k\in\N$. Then
    the iterates $(x_k)_{k\in\N_0}$ generated by \cref{alg:seqprox-sgd} with step size
    schedule $(\eta_k)_{k\in\N_0}$ satisfy
    \begin{align*}
        2\eta_k\E_k(f_k(x_{k+1}) - f_k(x))
        &\leq ( 1 - \mu\eta_k ) \norm{x - x_k}^2 - \E_k\norm{x - x_{k+1}}^2 + \frac{2\eta_k^2}{1 - \rho}\Var_k(g_k) \\
        &\phantom{===} + \frac{8 G^2\eta_k^2\gamma_k^2}{1 - \rho}.
    \end{align*}
    for all $x\in\R^n$ and $k\in\N_0$.
\end{lemma}
\begin{proof}
    For $k\in\N_0$ we denote by $g_k$ the stochastic gradient of $\psi_k$ at $x_k$
    that is used in iteration $k$ of \cref{alg:seqprox-sgd}.
    We also let $\psi(x) := \E(F_\xi(x))$ for $x\in\R^n$, so that
    $\psi_k = \psi + \gamma_k\pi_k$ and
    $\nabla \psi_k = \nabla \psi + \gamma_k \nabla\pi_k$.
    By $L$-smoothness of $\psi$ (\cref{ass:basic-assumptions}),
    we have
    \begin{align*}
        f_k(x_{k+1})
        &= \psi_k(x_{k+1}) + r(x_{k+1}) \\
        &= \psi(x_{k+1}) + \gamma_k\pi_k(x_{k+1}) + r(x_{k+1}) \\
        &\leq \psi(x_k) + \langle \nabla \psi(x_k), x_{k+1} - x_k \rangle + \frac{L}{2} \norm{x_{k+1} - x_k}^2 + \gamma_k\pi_k(x_{k+1}) + r(x_{k+1}) \\
        &= \psi(x_k) + \langle \nabla \psi_k(x_k), x_{k+1} - x_k \rangle + \frac{L}{2} \norm{x_{k+1} - x_k}^2
            + \gamma_k\pi_k(x_{k+1}) + r(x_{k+1}) \\
        &\phantom{===}+ \gamma_k \langle \nabla \pi_k(x_k), x_k - x_{k+1} \rangle.
    \end{align*}
    By convexity of $\pi_k$, we further have
    \[
        \pi_k(x_{k+1}) \leq \pi_k(x_k) + \langle \nabla \pi_k(x_{k+1}), x_{k+1} - x_k \rangle,
    \]
    and an application of Cauchy-Schwarz and Young's inequality yields
    \[
        \pi_k(x_{k+1})
        \leq \pi_k(x_k) + \frac{\epsilon_k^{-1}}{2} \norm{\nabla \pi_k(x_{k+1})}^2 + \frac{\epsilon_k}{2} \norm{x_{k+1} - x_k}^2,
    \]
    for any $\epsilon_k \in (0,\infty)$.
    Note that the gradients $(\nabla \pi_k)_{k\in\N_0}$ are bounded uniformly (\cref{lem:basic-properties}),
    hence there exists $G\in(0,\infty)$ such that
    \[
        \pi_k(x_{k+1})
        \leq \pi_k(x_k) + \frac{\epsilon_k}{2} G^2 + \frac{\epsilon_k^{-1}}{2} \norm{x_{k+1} - x_k}^2,
    \]
    for any $\epsilon_k\in(0,\infty)$.
    With this, we can further bound $f_k(x_{k+1})$ by
    \begin{align*}
        f_k(x_{k+1})
        &\leq \psi(x_k) + \langle \nabla \psi_k(x_k), x_{k+1} - x_k \rangle + \frac{L}{2} \norm{x_{k+1} - x_k}^2 + r(x_{k+1}) \\
        &\phantom{===}+ \gamma_k\Big( \pi_k(x_k) + \frac{\epsilon_k}{2} G^2 + \frac{\epsilon_k^{-1}}{2} \norm{x_{k+1} - x_k}^2 \Big)
            + \gamma_k \langle \nabla \pi_k(x_k), x_k - x_{k+1} \rangle. \\
        &= \psi_k(x_k) + \langle \nabla \psi_k(x_k), x_{k+1} - x_k \rangle + \frac{L + \gamma_k\epsilon_k^{-1}}{2} \norm{x_{k+1} - x_k}^2 + r(x_{k+1}) \\
        &\phantom{===} + \frac{\gamma_k \epsilon_k G^2}{2} + \gamma_k \langle \nabla \pi_k(x_k), x_k - x_{k+1} \rangle.
    \end{align*}
    By another application of Cauchy-Schwarz and Young's inequality, we have for any $\epsilon_k\in(0,\infty)$
    \[
         \langle \nabla \pi_k(x_k), x_k - x_{k+1} \rangle \leq \frac{\epsilon_k}{2} G^2 + \frac{\epsilon_k^{-1}}{2} \norm{x_{k+1} - x_k}^2,
    \]
    where again used the bound $\sup_{k\in\N_0}\sup_{x\in\R^n}\nabla \pi_k(x) \leq G$, and therefore we obtain
    \begin{align*}
        f_k(x_{k+1})
        &\leq \psi_k(x_k) + \langle \nabla \psi_k(x_k), x_{k+1} - x_k \rangle + \frac{L + 2\gamma_k\epsilon_k^{-1}}{2} \norm{x_{k+1} - x_k}^2 + r(x_{k+1}) \\
        &\phantom{===} + \gamma_k \epsilon_k G^2
    \end{align*}
    for any $\epsilon_k\in(0,\infty)$. We let $z_k := \nabla \psi_k(x_k) - g_k$ be the error in the $k$-th
    stochastic gradient. By adding and subtracting $\langle z_k, x_{k+1} - x_k \rangle$ in
    the above inequality, we get
    \begin{align*}
        f_k(x_{k+1})
        &\leq \psi_k(x_k) + \langle g_k, x_{k+1} - x_k \rangle + \frac{L + 2\gamma_k\epsilon_k^{-1}}{2} \norm{x_{k+1} - x_k}^2 + r(x_{k+1}) \\
        &\phantom{===} + \gamma_k \epsilon_k G^2 + \langle z_k, x_{k+1} - x_k \rangle.
    \end{align*}
    By yet another application of Cauchy-Schwarz and Young's inequality, we have
    \[
        \langle z_k, x_{k+1} - x_k \rangle
        \leq
        \frac{\delta_k}{2}\norm{z_k}^2 + \frac{\delta_k^{-1}}{2}\norm{x_{k+1} - x_k}^2,
    \]
    for all $\delta_k\in(0,\infty)$, and therefore
    \begin{align*}
        f_k(x_{k+1})
        &\leq \psi_k(x_k) + \langle g_k, x_{k+1} - x_k \rangle
            + \frac{L + 2\gamma_k\epsilon_k^{-1} + \delta_k^{-1}}{2} \norm{x_{k+1} - x_k}^2 + r(x_{k+1}) \\
        &\phantom{===} + \gamma_k \epsilon_k G^2 + \frac{\delta_k}{2} \norm{z_k}^2, \\
        &= \psi_k(x_k) + r(x_{k+1}) + \langle g_k, x_{k+1} - x_k \rangle + \frac{1}{2\eta_k}\norm{x_{k+1} - x_k}^2 \\
        &\phantom{===} + \frac{L + 2\gamma_k\epsilon_k^{-1} + \delta_k^{-1} - \eta_k^{-1}}{2} \norm{x_{k+1} - x_k}^2 + \gamma_k \epsilon_k G^2 + \frac{\delta_k}{2} \norm{z_k}^2,
    \end{align*}
    where in the last step we added and subtracted $(2\eta_k)^{-1} \norm{x_{k+1} - x_k}^2$ and moved $r(x_{k+1})$
    further forward.
    From the definition of the proximal operator, it follows that
    \begin{align*}
        x_{k+1} &= \prox_{\eta_k r} (x_k - \eta_k g_k) \\
                &= \argmin_{x\in\R^n} \Big\{\, r(x) + \frac{1}{2\eta_k}\norm{x - (x_k - \eta_k g_k)}^2 \,\Big\} \\
                &= \argmin_{x\in\R^n} \Big\{\, r(x) + \langle g_k, x - x_k \rangle + \frac{1}{2\eta_k}\norm{x - x_k}^2 \,\Big\},
    \end{align*}
    where the last step follows from expanding the square and dropping
    the constant term $\eta_k^2 \norm{g_k}^2$ from the minimization.
    The function $x\mapsto r(x) + \langle g_k, x - x_k \rangle + (2\eta_k)^{-1}\norm{x - x_k}^2$
    is $(2\eta_k)^{-1}$-strongly convex and minimized by $x_{k+1}$. Thus,
    comparing with our previous bound on $f_k(x_{k+1})$,
    we can conclude
    \begin{align*}
        f_k(x_{k+1})
        &\leq \psi_k(x_k) + r(x) + \langle g_k, x - x_k \rangle + \frac{1}{2\eta_k}\norm{x - x_k}^2
            - \frac{1}{2\eta_k}\norm{x - x_{k+1}}^2 \\
        &\phantom{===} + \frac{L + 2\gamma_k\epsilon_k^{-1} + \delta_k^{-1} - \eta_k^{-1}}{2} \norm{x_{k+1} - x_k}^2
            + \gamma_k \epsilon_k G^2 + \frac{\delta_k}{2} \norm{z_k}^2,
    \end{align*}
    for all $x\in\R^n$. By $\mu$-strong convexity of $\psi_k$, we have
    \begin{align*}
        \psi_k(x_k) + r(x) + \langle g_k, x - x_k \rangle
        &=
        \psi_k(x_k) + r(x) + \langle \nabla \psi_k(x_k), x - x_k \rangle + \langle z_k, x_k - x \rangle \\
        &\leq
        \psi_k(x) - \frac{\mu}{2}\norm{x - x_k}^2 + r(x) + \langle z_k, x_k - x \rangle \\
        &=
        f_k(x) - \frac{\mu}{2}\norm{x - x_k}^2 + \langle z_k, x_k - x \rangle,
    \end{align*}
    for all $x\in\R^n$. Hence,
    \begin{align}
        \label{eq:proof:lem:one-step-improvement}
        f_k(x_{k+1})
        &\leq f_k(x) + \Big( \frac{1}{2\eta_k} - \frac{\mu}{2} \Big) \norm{x - x_k}^2 + \langle z_k, x_k - x \rangle 
            - \frac{1}{2\eta_k}\norm{x - x_{k+1}}^2 \notag \\
        &\phantom{===} + \frac{L + 2\gamma_k\epsilon_k^{-1} + \delta_k^{-1} - \eta_k^{-1}}{2} \norm{x_{k+1} - x_k}^2
            + \gamma_k \epsilon_k G^2 + \frac{\delta_k}{2} \norm{z_k}^2.
    \end{align}
    Fix $\alpha \in (0, 1 - \rho)$ and define $\epsilon_k := 2\alpha^{-1}\eta_k\gamma_k$. Then
    \[
        (L + 2\gamma_k\epsilon_k^{-1})\eta_k
        = \Big( L + 2\gamma_k \frac{\alpha}{2 \eta_k \gamma_k} \Big)\eta_k
        = L\eta_k + \alpha \leq \rho + \alpha < 1,
    \]
    where we used that $\eta_k \leq \rho L^{-1}$, which holds per assumption.
    Choosing
    \[
        \delta_k := \frac{\eta_k}{1 - (L\eta_k + \alpha)} \in (0,\infty)
    \]
    therefore yields
    \[
        L + 2\gamma_k\epsilon_k^{-1} + \delta_k^{-1} - \eta_k^{-1}
        = L + \frac{\alpha}{\eta_k} - \frac{1 - (L\eta_k + \alpha)}{\eta_k} - \frac{1}{\eta_k}
        = 0.
    \]
    Hence, we can drop the $\norm{x_{k+1} - x_k}^2$ term from \eqref{eq:proof:lem:one-step-improvement} and get
    \begin{align*}
        f_k(x_{k+1})
        &\leq f_k(x) + \Big(\frac{1}{2\eta_k} - \frac{\mu}{2}\Big)\norm{x - x_k}^2 + \langle z_k, x_k - x \rangle 
            - \frac{1}{2\eta_k}\norm{x - x_{k+1}}^2 \\
        &\phantom{===} + 2 \alpha^{-1} \eta_k \gamma_k^2 G^2 + \frac{\eta_k}{2(1 - (L\eta_k + \alpha))}\norm{z_k}^2.
    \end{align*}
    Subtracting by $f_k(x)$ and multiplying both sides by $2\eta_k$
    yields
    \begin{align*}
        2\eta_k(f_k(x_{k+1}) - f_k(x))
        &\leq ( 1 - \mu\eta_k ) \norm{x - x_k}^2 + 2\eta_k\langle z_k, x_k - x \rangle 
            - \norm{x - x_{k+1}}^2 \\
        &\phantom{===} + 4 \alpha^{-1} \eta_k^2 \gamma_k^2 G^2 + \frac{\eta_k^2}{1 - (L\eta_k + \alpha)}\norm{z_k}^2.
    \end{align*}
    For the specifc choice $\alpha := (1 - \rho)/2$, it holds that $L\eta_k + \alpha \leq \rho + (1 - \rho)/2 = (1 + \rho)/2$.
    Hence, we arrive at
    \begin{align}
        \label{eq:proof:lem:one-step-improvement-2}
        2\eta_k(f_k(x_{k+1}) - f_k(x))
        &\leq ( 1 - \mu\eta_k ) \norm{x - x_k}^2 + 2\eta_k\langle z_k, x_k - x \rangle 
            - \norm{x - x_{k+1}}^2 \notag \\
        &\phantom{===} +  \frac{8 G^2\eta_k^2\gamma_k^2}{1 - \rho} + \frac{2\eta_k^2}{1 - \rho}\norm{z_k}^2.
    \end{align}
    By properties of conditional expectation and the definition of
    stochastic gradients, we know that $\E_k \langle z_k, x_k - x \rangle = \langle \E_k(z_k), x_k - x \rangle = 0$, and
    thus, applying conditional expectations to both sides of the above
    inequality, we obtain
    \begin{align*}
        2\eta_k\E_k(f_k(x_{k+1}) - f_k(x))
        &\leq ( 1 - \mu\eta_k ) \norm{x - x_k}^2 - \E_k\norm{x - x_{k+1}}^2 \\
        &\phantom{===} + \frac{8 G^2\eta_k^2\gamma_k^2}{1 - \rho} + \frac{2\eta_k^2}{1 - \rho}\E_k\norm{z_k}^2.
    \end{align*}
    for all $x\in\R^n$. Since $\E_k\norm{z_k}^2 = \E_k\norm{g_k - \nabla \psi_k(x_k)}^2$ and
    $\E_k(g_k) = \nabla \psi_k(x_k)$ per definition of $g_k$, it holds that $\E_k\norm{z_k}^2 = \Var_k(g_k)$.
    We have thus arrived at the desired result.
\end{proof}

After rearranging the inequality from \cref{lem:one-step-improvement}, setting $x = x^\star$, and
dropping the $-\mu\eta_k$ term, we get
\begin{equation}
    \label{eq:sec:almost-sure-convergence}
    \E_k\norm{x^\star - x_{k+1}}^2
    \leq
    \norm{x^\star - x_k}^2
    + \frac{2\eta_k^2}{1 - \rho}\Var_k(g_k) + \frac{8 G^2\eta_k^2\gamma_k^2}{1 - \rho}
    - 2\eta_k\E_k(f_k(x_{k+1}) - f_k(x^\star)).
\end{equation}
This is not yet quite in the form needed to apply \cref{lem:robbins-siegmund}. For one,
we need to bound the noise term $\Var_k(g_k)$. Second, we cannot
in general guarantee that $\E_k(f_k(x_{k+1}) - f_k(x^\star)) \geq 0$.
However, as we show in the next lemma, we can find a lower bound that
involves a nonnegative term plus a small negative term.
% + \frac{2\eta_k^2}{1 - \rho}\Var_k(g_k) \\
%         &\phantom{===} + \frac{8 G^2\eta_k^2\gamma_k^2}{1 - \rho}
\begin{lemma}
    \label{lem:recursion-for-almost-sure-convergence}
    Assume that \cref{ass:basic-assumptions,ass:slater-point,ass:compact-support-without-zero} hold, and
    let $\eta_k\in(0,\rho L^{-1}]$ for all $k\in\N_0$ and some $\rho\in(0,1)$. Then the iterates
    $(x_k)_{k\in\N_0}$ generated
    by \cref{alg:seqprox-sgd} with step size schedule $(\eta_k)_{k\in\N_0}$ satisfy
    \begin{align*}
        \E_k\norm{x^\star - x_{k+1}}^2
        &\leq
        (1-\mu\eta_k)\norm{x^\star - x_k}^2
        + \frac{2\eta_k^2}{1 - \rho}\Var_k(g_k) + \frac{8 G^2\eta_k^2\gamma_k^2}{1 - \rho} \\
        &\phantom{===} + \mathcal{O}(\eta_k\gamma_k\alpha_k)
        - 2\eta_k\E_k(f_{k}(x_{k+1}) - f_{k}(x_k^\star)),
    \end{align*}
    for all $k\in\N_0$.
\end{lemma}
\begin{proof}
    Let $k\in\N_0$. From \cref{lem:one-step-improvement}, we have
    \begin{align*}
        2\eta_k\E_k(f_k(x_{k+1}) - f_k(x^\star))
        &\leq
            (1-\mu\eta_k)\norm{x^\star - x_k}^2
            - \E_k\norm{x^\star - x_{k+1}}^2
            + \frac{2\eta_k^2}{1 - \rho}\Var_k(g_k)  \\
        &\phantom{===} + \frac{8 G^2\eta_k^2\gamma_k^2}{1 - \rho}.
    \end{align*}
    Write
    \begin{align}
        \label{eq:proof:lem:recursion-for-almost-sure-convergence}
        f_k(x_{k+1}) - f_k(x^\star)
        &=
       (f_k(x_{k+1}) - f_k(x_k^\star)) + (f_k(x_k^\star) - f_k(\proj(x_k^\star))) \\
        &\phantom{===}+ (f_k(\proj(x_k^\star)) - f_k(x^\star)). \notag
    \end{align}
    By definition, it holds that
    \begin{align*}
        f_k(\proj(x_k^\star)) - f_k(x^\star)
        &=
        f(\proj(x_k^\star)) + \gamma_k \pi_k(\proj(x_k^\star)) - f(x^\star) - \gamma_k \pi_k(x^\star) \\
        &\geq
        f(\proj(x_k^\star)) - f(x^\star) - \gamma_k \pi_k(x^\star) \\
        &\geq
        - \gamma_k \pi_k(x^\star),
    \end{align*}
    where the last step follows from the fact that $x^\star$ minimizes $f$ on $\mathcal{X}$
    and $\proj(x_k^\star)\in\mathcal{X}$. Further, feasibility implies $\pi_k(x^\star) \leq \alpha_k$.
    Hence, combining with \eqref{eq:proof:lem:recursion-for-almost-sure-convergence}, we have
    \[
        f_k(x_{k+1}) - f_k(x^\star) \geq (f_k(x_{k+1}) - f_k(x_k^\star)) + f_k(x_k^\star) - f_k(\proj(x_k^\star)) - \gamma_k\alpha_k.
    \]
    For the next steps, we let $\tilde{\nabla} f(x)$ denote the minimum-norm subgradient
    in $\partial f(x)$, for all $x\in\dom(f)$.
    To analyze $f_k(x_k^\star) - f_k(\proj(x_k^\star))$, we first use convexity and Cauchy-Schwarz to get
    \[
        f_k(x_k^\star) - f_k(\proj(x_k^\star))
        \geq \langle \tilde{\nabla} f_k(\proj(x_k^\star)), x_k^\star - \proj(x_k^\star) \rangle
        \geq -\norm{\tilde{\nabla} f_k(\proj(x_k^\star))} \dist(x_k^\star, \mathcal{X}).
    \]
    The sequence $(x_k^\star)_{k\in\N}$ converges to $x^\star$, which implies, by continuity
    of the projection map, $\lim_{k\to\infty} \proj(x_k^\star) = \proj(x^\star) = x^\star$.
    In particular, $(\proj(x_k^\star))_{k\in\N}$ is bounded, so \cref{lem:basic-properties} implies
    that there exists $K_1\in\N$ such that $\sup_{k\in\N} \norm{\tilde{\nabla} f_k(\proj(x_k^\star))}
    \leq c \gamma_k$ for some $c_1\in(0,\infty)$ and all $k\geq K_1$. Hence,
    \[
        f_k(x_k^\star) - f_k(\proj(x_k^\star)) \geq -c_1 \gamma_k\, \dist(x_k^\star,\mathcal{X}),
    \]
    for all $k\geq K_1$.
    By \cref{thm:distance-to-feasibility-bound}, it holds that there exist
    $K_2\in\N$ and $c_2\in(0,\infty)$ such that, for all $k\geq K_2$, $\dist(x_k^\star,\mathcal{X}) \leq c_2 \, \alpha_k$.
    Setting $K := \max(K_1, K_2)$, it therefore holds that
    \[
        f_k(x_k^\star) - f_k(\proj(x_k^\star)) \geq -c_1c_2 \cdot \gamma_k \alpha_k
    \]
    and
    \begin{align*}
        f_k(x_{k+1}) - f_k(x^\star)
        &\geq
        f_k(x_{k+1}) - f_k(x_k^\star)
        -c_1c_2\cdot \gamma_k\alpha_k - \gamma_k\alpha_k \\
        &=
        f_k(x_{k+1}) - f_k(x_k^\star)
        -(1 + c_1c_2) \, \gamma_k\alpha_k,
    \end{align*}
    for all $k\geq K$.
    Plugging this into our original estimate, we have
    % + \frac{2\eta_k^2}{1 - \rho}\Var_k(g_k) \\
    %         &\phantom{===} + \frac{8 G^2\eta_k^2\gamma_k^2}{1 - \rho}
    \begin{align*}
        2\eta_k\big(\E_k(f_k(x_{k+1}) - f_k(x_k^\star))
        - (1 + c_1c_2)\gamma_k\alpha_k\big)
        &\leq
        (1-\mu\eta_k)\norm{x^\star - x_k}^2
        - \E_k\norm{x^\star - x_{k+1}}^2 \\
        &\phantom{===} + \frac{2\eta_k^2}{1 - \rho}\Var_k(g_k) + \frac{8 G^2\eta_k^2\gamma_k^2}{1 - \rho},
    \end{align*}
    and thus
    \begin{align*}
        \E_k\norm{x^\star - x_{k+1}}^2
        &\leq
        (1-\mu\eta_k)\norm{x^\star - x_k}^2
        + \frac{2\eta_k^2}{1 - \rho}\Var_k(g_k) + \frac{8 G^2\eta_k^2\gamma_k^2}{1 - \rho} \\
        &\phantom{===} + 2(1 + c_1c_2)\,\eta_k\gamma_k\alpha_k
        - 2\eta_k\E_k(f_k(x_{k+1}) - f_k(x_k^\star)),
    \end{align*}
    for all $k\geq K$, which directly implies the claim.
\end{proof}
Combining the above lemma with \eqref{eq:sec:almost-sure-convergence}, we now have
\begin{align*}
    \E_k\norm{x^\star - x_{k+1}}^2
    &\leq
    \norm{x^\star - x_k}^2
    + \frac{2\eta_k^2}{1 - \rho}\Var_k(g_k) + \frac{8 G^2\eta_k^2\gamma_k^2}{1 - \rho}
    + \mathcal{O}(\eta_k\gamma_k\alpha_k) \\
    &\phantom{===}- 2\eta_k\E_k(f_k(x_{k+1}) - f_k(x_k^\star)).
\end{align*}
The term $2\eta_k\E_k(f_k(x_{k+1}) - f_k(x_k^\star))$ is indeed nonnegative by optimality of $x_k^\star$
for $f_k$. Assuming that $\sum_{k=0}^\infty \eta_k\gamma_k\alpha_k < \infty$ and
$\sum_{k=0}^\infty \eta_k^2 \gamma_k^2 < \infty$, we are almost ready
to apply \cref{lem:robbins-siegmund}.
The following lemma will give a bound on the gradient noise $\Var_k(g_k)$.
\begin{lemma}[Bound on gradient noise]
    \label{lem:bound-on-variance}
    Let \cref{ass:basic-assumptions} hold. Then, in the situation of \cref{alg:seqprox-sgd}, there
    exists a constant $\sigma^2\in(0,\infty)$ such that
    \[
        \Var_k(g_k) \leq \frac{2L^2}{\beta_k} \norm{x_k - x^\star}^2 + \frac{1 + \gamma_k^2}{\beta_k} \sigma^2,
    \]
    for all $k\in\N_0$.
\end{lemma}
\begin{proof}
    Let $k\in\N_0$ and define
    $\tilde{g}_k := \nabla F_{\xi_k}(x_k) + \gamma_k \nabla h_k(x_k; A(\xi_k), b(\xi_k))$.
    We have
    \begin{align*}
        \Var_k(g_k)
        &=
        \frac{1}{\beta_k} \Var_k(\tilde{g}_k) \\
        &\leq
        \frac{1}{\beta_k} \E_k\norm{\tilde{g}_k}^2 \\
        &\leq
        \frac{1}{\beta_k} \Big( \E_k\norm{\nabla F_\xi(x_k)}^2
            + \gamma_k^2\, \E_k\norm{\nabla h_k(x_k; A(\xi), b(\xi))}^2 \Big).
    \end{align*}
    Using the inequality $(a + b)^2 \leq 2a^2 + 2b^2 \enspace\forall a,b\in\R$, and the (almost sure)
    $L$-smoothness of $x\mapsto F_\xi(x)$, we have
    \begin{align*}
        \E_k\norm{\nabla F_\xi(x_k)}^2
        &=
        \E_k\norm{\nabla F_\xi(x_k) - \nabla F_\xi(x^\star) + \nabla F_\xi(x^\star)}^2 \\
        &\leq
        2\Big( \E_k\norm{\nabla F_\xi(x_k) - \nabla F_\xi(x^\star)}^2 + \E\norm{\nabla F_\xi(x^\star)}^2 \Big) \\
        &\leq 2 L^2\, \norm{x_k - x^\star}^2 + 2\,\E\norm{\nabla F_\xi(x^\star)}^2,
    \end{align*}
    where in the last step we also used that $x_k$ is $\mathcal{F}_k$-measurable and
    $\xi$ is independent of $\mathcal{F}_k$.
    By one of our assumptions, we can find a point $x\in\R^n$ such that
    $\E\norm{F_\xi(x)}^2 < \infty$. Hence, using smoothness and the inequality
    $(a + b)^2 \leq 2a^2 + 2b^2 \enspace\forall a,b\in\R$ again, there exists a constant
    $M^2\in(0,\infty)$ such that
    \begin{align*}
        \E\norm{\nabla F_\xi(x^\star)}^2
        &=
        \E\norm{\nabla F_\xi(x^\star) - \nabla F_\xi(x) + \nabla F_\xi(x)}^2 \\
        &\leq
        2L^2\,\norm{x^\star - x}^2 + 2\,\E\norm{F_\xi(x)}^2 \\
        &\leq
        \frac{1}{2}M^2.
    \end{align*}
    Therefore, combining with the previous inequality, we get the bound
    \[
        \E_k\norm{\nabla F_\xi(x_k)}^2 \leq 2 L^2\, \norm{x_k - x^\star}^2 + M^2.
    \]
    Since, per \cref{ass:basic-assumptions}, it holds that the family $(h_k)_{k\in\N}$ has uniformly bounded
    second moment,
    there exists a constant
    $\tilde{M}^2\in[0,\infty)$ such that
    \[
        \E_k\norm{\nabla h_k(x_k; A(\xi), b(\xi))}^2 \leq \tilde{M}^2.
    \]
    Putting everything together, we obtain
    \[
        \Var_k(g_k)
        \leq \frac{1}{\beta_k} \Big( 2 L^2\, \norm{x_k - x^\star}^2 + M^2
            + \gamma_k^2 \tilde{M}^2 \Big)
        = \frac{2L^2}{\beta_k} \norm{x_k - x^\star}^2
            + \frac{1}{\beta_k} (M^2 + \gamma_k^2 \tilde{M}^2).
    \]
    Setting $\sigma^2 := \max(M^2, \tilde{M}^2)$ yields the desired result.
\end{proof}

We are now ready to prove the main theorem of this section.
\begin{theorem}[Almost sure convergence]
    \label{thm:almost-sure-convergence}
    Assume \cref{ass:basic-assumptions,ass:slater-point,ass:compact-support-without-zero} hold.
    Let $(x_k)_{k\in\N_0}$ be a sequence generated by \cref{alg:seqprox-sgd} with
    parameters $(\eta_k)_{k\in\N_0}$, $(\gamma_k)_{k\in\N_0}$, $(h_k)_{k\in\N_0}$ that satisfiy
    \begin{enumerate}
        \item $\sum_{k=0}^\infty \eta_k = \infty$ and there exists $\rho\in(0,1)$ such that $\eta_k \leq \rho L^{-1}$ for all $k\in\N$.
        \item $\sum_{k=0}^\infty \eta_k\gamma_k\alpha_k < \infty$.
        \item $\sum_{k=0}^\infty \eta_k^2 \gamma_k^2 < \infty$.
    \end{enumerate}
    Then $\norm{x_{k} - x^\star}$ converges almost surely and
    $\liminf_{k\to\infty} \E\norm{x_{k} - x^\star}^2 = 0$.
    In particular, $(x_k)_{k\in\N}$ is bounded almost surely.
\end{theorem}
\begin{proof}
    By \cref{lem:recursion-for-almost-sure-convergence} (dropping the $-\mu\eta_k$ term), we have
    \begin{align*}
        \E_k\norm{x^\star - x_{k+1}}^2
        &\leq
        \norm{x^\star - x_k}^2
        + \frac{2\eta_k^2}{1 - \rho}\Var_k(g_k) + \frac{8 G^2\eta_k^2\gamma_k^2}{1 - \rho} \\
        &\phantom{===} + \mathcal{O}(\eta_k\gamma_k\alpha_k)
        - 2\eta_k\E_k(f_{k}(x_{k+1}) - f_{k}(x_k^\star)),
    \end{align*}
    \Cref{lem:bound-on-variance} lets us bound
    the variance by
    \[
        \Var_k(g_k) \leq \frac{2L^2}{\beta_k} \norm{x_k - x^\star}^2 + \frac{1 + \gamma_k^2}{\beta_k} \sigma^2,
    \]
    for a constant $M^2\in(0,\infty)$ and all $k\in\N_0$.
    Thus, we have
    \begin{align*}
        \E_k\norm{x^\star - x_{k+1}}^2
        &\leq
        \norm{x^\star - x_k}^2
        + \frac{2\eta_k^2}{1-\rho}\Big( \frac{2L^2}{\beta_k} \norm{x_k - x^\star}^2 + \frac{1 + \gamma_k^2}{\beta_k} \sigma^2 \Big) \\
        &\phantom{===} + \mathcal{O}(\eta_k\gamma_k\alpha_k) - 2\eta_k\E_k(f_{k}(x_{k+1}) - f_{k}(x_k^\star))\\
        &=
        \Big( 1 + 4L^2(1-\rho)^{-1}\beta_k^{-1}\eta_k^2 \Big)\norm{x^\star - x_k}^2
        + 2\eta_k^2(1-\rho)^{-1}\frac{1 + \gamma_k^2}{\beta_k} \sigma^2 \\
        &\phantom{===} + \mathcal{O}(\eta_k\gamma_k\alpha_k) - 2\eta_k\E_k(f_{k}(x_{k+1}) - f_{k}(x_k^\star)),
    \end{align*}
    for all $k\in\N_0$.
    Define the nonegative sequences $(a_k)_{k\in\N_0}, (b_k)_{k\in\N_0}, (c_k)_{k\in\N_0}$ by
    \begin{align*}
        a_k &:= 4L^2(1-\rho)^{-1}\beta_k^{-1}\eta_k^2 \\
        b_k &:= 2\eta_k^2(1-\rho)^{-1}\frac{1 + \gamma_k^2}{\beta_k} \sigma^2 + \mathcal{O}(\eta_k\gamma_k\alpha_k)\ \\
        c_k &:= 2\eta_k\E_k(f_{k}(x_{k+1}) - f_{k}(x_k^\star)).
    \end{align*}
    Note that $c_k$ is indeed nonnegative, since $x_k^\star$ minimizes $f_k$.
    The above inequality now takes the form
    \[
        \E_k\norm{x^\star - x_{k+1}}^2
        \leq
        (1 + a_k)\norm{x^\star - x_k}^2
        + b_k - c_k.
    \]
    Our assumptions imply that $\sum_{k=0}^{\infty} a_k < \infty$ and $\sum_{k=0}^\infty b_k < \infty$,
    so we can apply
    \cref{lem:robbins-siegmund}, which implies that with probability one the sequence
    $\norm{x^\star - x_{k}}^2$ converges and
    $\sum_{k=0}^\infty \eta_k\E_k(f_{k}(x_{k+1}) - f_{k}(x_k^\star)) < \infty$.
    By the bounded convergence theorem (\textcolor{red}{TODO: add ref}), it further holds that
    \begin{align*}
        \infty > \E\Big( \sum_{k=0}^\infty \eta_k\E_k(f_{k}(x_{k+1}) - f_{k}(x_k^\star)) \Big)
        &=
        \sum_{k=0}^\infty \eta_k\E(\E_k(f_{k}(x_{k+1}) - f_{k}(x_k^\star))) \\
        &=
        \sum_{k=0}^\infty \eta_k\E(f_{k}(x_{k+1}) - f_{k}(x_k^\star)).
    \end{align*}
    Since $\sum_{k=0}^\infty \eta_k = \infty$, it must therefore hold that
    \[
        \liminf_{k\to\infty} \E(f_{k}(x_{k+1}) - f_{k}(x_k^\star)) = 0.
    \]
    Strong convexity of $f_k$ and optimaliy of $x_k^\star$ for $f_k$ imply
    \[
        f_{k}(x_{k+1}) - f_{k}(x_k^\star) \geq \frac{\mu}{2} \norm{x_{k+1} - x_k^\star}^2,
    \]
    and thus $\liminf_{k\to\infty} \E\norm{x_{k+1} - x_k^\star}^2 = 0$,
    which implies that
    \[
        \liminf_{k\to\infty} \E\norm{x_{k} - x^\star}^2 = 0,
    \]
    since $x_k^\star$ converges to $x^\star$ (\cref{thm:distance-to-feasibility-bound}).
\end{proof}


\section{Convergence rates in expectation}

In the previous section, we established that the iterates $(x_k)_{k\in\N_0}$
of \cref{alg:seqprox-sgd} are bounded with probability one,
provided that the parameters $(h_k)_{k\in\N_0}$, $(\eta_k)_{k\in\N_0}$, $(\gamma_k)_{k\in\N_0}$
satisfy certain conditions, which are captured in the following assumption.

% \begin{theorem}[Constant step size schedule]
%     For all $k\in\N_0$, let $\eta_k \equiv \eta \in(0, 1/(2L)]$, $\gamma_k = \mathcal{O}(\log^e(1+k))$ for any $e\in(0, \infty)$,
%     and $\beta_k := 8L\mu^{-1}(1+\gamma_k^2)$. Then, the iterates $(x_k)_{k\in\N}$ generated
%     by \cref{alg:seqprox-sgd} with parameters $(\eta_k)_{k\in\N}$, $\gamma_k = \mathcal{O}(\log^e(k))$,
%     $(\beta_k)_{k\in\N}$, and $(h_k)_{k\in\N}$ such that $\alpha_k = \mathcal{O}(k^{-a})$ for some $a\in(0,\infty)$,
%     satisfy
%     \[
%         \E\norm{x_{k} - x_{k}^\star}^2
%         \leq ( 1 - \mu\eta / 2 )^k \, \E\norm{x_{0} - x_{0}^\star}^2
%             + \eta^2 \frac{\mu \sigma^2}{2 L}
%     \]
% \end{theorem}

\begin{assumption}
    \label{ass:step-sizes-and-penalties}
    The parameters
    $(h_k)_{k\in\N_0}$, $(\eta_k)_{k\in\N_0}$, $(\gamma_k)_{k\in\N_0}$ of \cref{alg:seqprox-sgd} satisfiy
    \begin{enumerate}
        \item $\sum_{k=0}^\infty \eta_k = \infty$ and there exist constants $\rho\in(0,1)$ and $K\in\N$
            such that $\eta_k \leq \rho L^{-1}$ for all $k\in\N, k\geq K$.
        \item $\sum_{k=0}^\infty \eta_k\gamma_k\alpha_k < \infty$.
        \item $\sum_{k=0}^\infty \eta_k^2 \gamma_k^2 < \infty$.
    \end{enumerate}
\end{assumption}

\begin{lemma}
    \label{lem:functional-upper-bound}
    Let \cref{ass:basic-assumptions,ass:compact-support-without-zero,ass:slater-point,ass:step-sizes-and-penalties} hold.
    Then there exists a deterministic sequence $(M_k^2)_{k\in\N_0} \subset \R^n$ with
    $M_k^2 = \mathcal{O}(\beta_k^{-1}(1 + \gamma_k^2) + \gamma_k^2)$, such that
    \[
        2\eta_k\E(f_k(x_{k}) - f_k(x))
        \leq
        (1-\mu\eta_k)\E\norm{x - x_k}^2 - \E \norm{x - x_{k+1}}^2 + \eta_k^2 M_k^2,
    \]
    for all $k\in\N_0$ and $x\in\R^n$.
\end{lemma}
\begin{proof}
    Let $k\in\N_0$. Taking expectations on both sides of the inequality provided by
    \cref{lem:one-step-improvement}, we get
    \begin{align*}
        2\eta_k\E(f_k(x_{k+1}) - f_k(x))
        &\leq ( 1 - \mu\eta_k ) \E\norm{x - x_k}^2 - \E\norm{x - x_{k+1}}^2 + \frac{2\eta_k^2}{1 - \rho}\E(\Var_k(g_k)) \\
        &\phantom{===} + \frac{8 G^2\eta_k^2\gamma_k^2}{1 - \rho},
    \end{align*}
    for all $x\in\R^n$. Recall that \cref{lem:bound-on-variance} lets us bound the conditional
    variance $\Var_k(g_k)$ as follows:
    \[
        \Var_k(g_k) \leq \frac{2L^2}{\beta_k} \norm{x_k - x^\star}^2 + \frac{1 + \gamma_k^2}{\beta_k}\sigma^2,
    \]
    for a constant $\sigma^2\in(0,\infty)$.
    \Cref{ass:step-sizes-and-penalties} and \cref{thm:almost-sure-convergence} imply that there exists
    a constant $\tilde{\sigma}^2\in(0, \infty)$ such that $2L^2 \norm{x_k - x^\star}^2 \leq \tilde{\sigma}^2$
    for all $k\in\N_0$ (almost surely).  We can assume without loss of generality that $\sigma^2 \leq \tilde{\sigma}^2$,
    and since $1 + \gamma_k^2 > 1$, we have
    \[
        \Var_k(g_k) \leq \frac{1}{\beta_k} \tilde{\sigma}^2 + \frac{1 + \gamma_k^2}{\beta_k}\tilde{\sigma}^2
        \leq \frac{2(1 + \gamma_k^2)}{\beta_k}\tilde{\sigma}^2,
    \]
    almost surely. Therefore, we obtain
    \[
        2\eta_k\E(f_k(x_{k+1}) - f_k(x))
        \leq
        (1-\mu\eta_k)\E\norm{x - x_k}^2
        - \E \norm{x - x_{k+1}}^2
        + \frac{4\eta_k^2 (1 + \gamma_k^2)}{(1-\rho)\beta_k}\tilde{\sigma}^2,
    \]
    for all $x\in\R^n$. Let $\tilde{\nabla} f_k(x_k)$ denote the minimum-norm subgradient
    of $f_k$ at $x_k$. By convexity of $f_k$ and Cauchy-Schwarz, we have
    \[
        f_k(x_{k+1})
        \geq
        f_k(x_k) + \langle \tilde{\nabla} f_k(x_k), x_{k+1} - x_k \rangle
        \geq
        f_k(x_k) - \norm{\tilde{\nabla} f_k(x_k)} \norm{x_{k+1} - x_k},
    \]
    hence
    \begin{align*}
        2\eta_k\E(f_k(x_{k}) - f_k(x))
        &\leq
        (1-\mu\eta_k)\E\norm{x - x_k}^2 - \E \norm{x - x_{k+1}}^2 + \frac{4\eta_k^2 (1 + \gamma_k^2)}{(1-\rho)\beta_k}\tilde{\sigma}^2 \\
        &\phantom{===}+ 2\eta_k\E\left( \norm{\tilde{\nabla} f_k(x_k)} \norm{x_{k+1} - x_k} \right).
    \end{align*}
    Since $(x_k)_{k\in\N_0}$ is bounded almost surely (\cref{thm:almost-sure-convergence}),
    we know from \cref{lem:basic-properties} that $\norm{\tilde{\nabla} f_k(x_k)} = \mathcal{O}(\gamma_k)$.
    We will now analyze $\norm{x_{k+1} - x_k}$. Let $y_k := \prox_{\eta_k r} (x_k)$. Then, using
    the triangle inequality and the nonexpansiveness property of the proximal operator (\textcolor{red}{TODO}),
    we have
    \begin{align*}
        \norm{x_{k+1} - x_k} &= \norm{x_{k+1} - y_k + y_k - x_k} \\
                            &\leq \norm{\prox_{\eta_k r}(x_k - \eta_k g_k) - \prox_{\eta_k r}(x_k)} + \norm{y_k - x_k} \\
                            &\leq \eta_k \norm{g_k} + \norm{y_k - x_k}.
    \end{align*}
    By \cref{lem:basic-properties}, for the stochastic gradient it holds that $\norm{g_k} = \mathcal{O}(\gamma_k)$
    almost surely. For the second term, we use the definition
    of $y_k$ as the solution to $\min_{x\in\R^n} \{ r(x) + (2\eta_k)^{-1} \norm{x - x_k}^2 \}$.
    By the first-order optimality condition, there exists a subgradient $\tilde{\nabla} r(y_k) \in \partial r(y_k)$ such that
    \[
        \tilde{\nabla} r(y_k) + \frac{y_k - x_k}{\eta_k} = 0 \enspace\iff\enspace \frac{x_k - y_k}{\eta_k} \in \partial r(y_k).
    \]
    Since $x_k$ is in the domain of $r$ \textcolor{red}{(TODO)}, we can now use the above, together with convexity, to get
    \[
        r(x_k) - r(y_k) \geq \langle \eta_k^{-1} (x_k - y_k), x_k - y_k \rangle = \eta_k^{-1} \norm{x_k - y_k}^2.
    \]
    Finally, local Lipschitz continuity of $r$  (\cref{ass:basic-assumptions}) yields (a.\,s.)
    \[
        L_r \norm{x_k - y_k} \geq r(x_k) - r(y_k) \geq \eta_k^{-1} \norm{x_k - y_k}^2
        \enspace\iff\enspace
        \norm{x_k - y_k} \leq \eta_k L_r,
    \]
    for some $L_r \in (0,\infty)$, where we again used that $(x_k)_{k\in\N_0}$ is bounded almost surely
    (\cref{thm:almost-sure-convergence})
    and the prox operator is nonexpansive (\textcolor{red}{TODO}).
    In total, we have thus shown
    \[
        \norm{x_k - x_{k+1}} = \mathcal{O}(\eta_k \gamma_k).
    \]
    It follows that
    \begin{align*}
        2\eta_k\E(f_k(x_{k}) - f_k(x))
        &\leq
        (1-\mu\eta_k)\E\norm{x - x_k}^2 - \E \norm{x - x_{k+1}}^2 + \frac{4\eta_k^2 (1 + \gamma_k^2)}{(1-\rho)\beta_k}\tilde{\sigma}^2 \\
        &\phantom{===}+ 2\eta_k^2\mathcal{O}(\gamma_k^2).
    \end{align*}
    Therefore, there exists a sequence $(M_k^2)_{k\in\N_0} \subset \R_{\geq 0}$ such that
    $M_k^2 = \mathcal{O}(\beta_k^{-1}(1 + \gamma_k^2) + \gamma_k^2)$ and
    \[
        2\eta_k\E(f_k(x_{k}) - f_k(x))
        \leq
        (1-\mu\eta_k)\E\norm{x - x_k}^2 - \E \norm{x - x_{k+1}}^2 + \eta_k^2 M_k^2,
    \]
    for all $x\in\R^n, k\in\N_0$, as desired.
\end{proof}

\begin{lemma}
    \label{lem:valid-parameters}
    Let $\eta_k := 2/(\mu k)$ for all $k\in\N$, $\eta_0 := 2/\mu$. Then, for any $e\in(0,\infty)$, there exist sequences
    $(\gamma_k)_{k\in\N_0}$ and
    $(h_k)_{k\in\N_0}$
    such that $\gamma_k = \mathcal{O}(\log^{e}(k))$,
    $\alpha_k = \mathcal{O}(1/k)$
    and \cref{ass:step-sizes-and-penalties} holds.
\end{lemma}
\begin{proof}
    
\end{proof}

\begin{lemma}
    \label{lem:upper-bound-on-f(xk)-f(xstar)-and-dist(xk)}
    Let \cref{ass:basic-assumptions,ass:compact-support-without-zero,ass:slater-point}
    hold and define $\eta_k := 2/(\mu k)$ for all $k\in\N$, $\eta_0 := 2/\mu$.
    Then there exists a constant $k_0\in\N$, as well as sequences $(M_k^2)_{k\in\N_0}\subset\R^n$,
    $(\beta_k)_{k\in\N_0}$, $(\gamma_k)_{k\in\N_0}$, $(h_k)_{k\in\N_0}$
    such that the iterates $(\bar{x}_k)_{k\in\N}$ generated by \cref{alg:seqprox-sgd}
    with parameters $(\eta_k)_{k\in\N_0}$, $(\gamma_k)_{k\in\N_0}$, $(h_k)_{k\in\N_0}$, $(\beta_k)_{k\in\N_0}$
    satisfy
    \[
        \E(f(\bar{x}_{K}) - f(x^\star))
        \leq
        \frac{S_{1, k_0-1}}{S_K} \E(f(\bar{x}_{1, k_0-1}) - f(x^\star))
        +
        \frac{e_{k_0}}{2S_{K}} + \frac{\sum_{k=k_0}^{K} M_k^2}{2S_{K}} + \frac{\sum_{k=k_0}^{K}\eta_k^{-1}\gamma_k\alpha_k}{S_{K}},
    \]
    for all $K\in\N$, where $S_K := \sum_{k=1}^{K} \eta_k^{-1}$,
    $e_{k_0} := \eta_{k_0-1}^{-2} \E\norm{x^\star - x_{k_0}}^2$ and
    $M_k^2 = \mathcal{O}(\beta_k^{-1}(1 + \gamma_k^2) + \gamma_k^2)$.
    Furthermore, there exist constants $k_1\in\N$ and $\tau \in (0, \infty)$ such that
    \begin{align*}
        \E(\dist(\bar{x}_K, \mathcal{X}))
        &\leq
        \frac{S_{1, k_1-1}}{S_K} \E(\dist(\bar{x}_{1, k_1-1}))
            + \frac{\tau^{-1}d_{k_1}}{S_{K}}
            + \frac{\tau^{-1}\sum_{k=k_1}^{K} \gamma_k^{-1}M_k^2}{S_{K}} \\
        &\phantom{===}+ \frac{2\tau^{-1}\sum_{k=k_1}^{K}\eta_k^{-1}\alpha_k}{S_{K}},
    \end{align*}
    for all $K\in\N$ with $K \geq k_1$, where $d_{k_1} := \gamma_{k_1}^{-1} \eta_{k_1-1}^{-2} \E(\dist(x_{k_1}, \mathcal{X})^2)$.
\end{lemma}

\begin{proof}
    Let $k_0\in\N$ be large enough such that $2/(\mu k_0) \leq 1/(2L)$ and fix
    some $k\in\N$ with $k \geq k_0$.
    \Cref{lem:valid-parameters} implies that there exist parameters $(\gamma_k)_{k\in\N_0}$ and
    $(h_k)_{k\in\N_0}$ such that \cref{ass:step-sizes-and-penalties} holds.
    Thus we can apply \cref{lem:functional-upper-bound}, which implies
    \[
        2\eta_k\E(f_k(x_{k}) - f_k(x))
        \leq
        (1-\mu\eta_k)\E\norm{x - x_k}^2 - \E \norm{x - x_{k+1}}^2 + \eta_k^2 M_k^2,
    \]
    for any $x\in\R^n$.
    Assume now that $x\in\mathcal{X}$. By a property of $\pi_k$, the fact that
    $(x_k)_{k\in\N}$ is bounded almost surely (\cref{thm:almost-sure-convergence}), and \cref{lem:lower-bound-on-penalties-by-distance-to-feasibility},
    we can deduce that there exists $\tau\in(0,\infty)$ such that
    \begin{align*}
        f_k(x_{k}) - f_k(x)
        &= f(x_k) - f(x) + \gamma_k (\pi_k(x_k) - \pi_k(x)) \\
        &\geq f(x_k) - f(x) + \gamma_k (\tau\dist(x_k,\mathcal{X}) - \alpha_k)
    \end{align*}
    with probability one. Hence, we have
    \begin{align}
        \label{eq:lem:upper-bound-on-f(xk)-f(xstar)-and-dist(xk)}
        2\eta_k \E(f(x_{k}) - f(x))
        &\leq
        (1-\mu\eta_k)\E\norm{x - x_k}^2 - \E \norm{x - x_{k+1}}^2
            + \eta_k^2 M_k^2 + 2\eta_k\gamma_k\alpha_k \notag \\
            &\phantom{===} - 2\tau \eta_k \gamma_k \E(\dist(x_k, \mathcal{X})),
    \end{align}
    for all $x\in\mathcal{X}$.
    We can now closely follow the proof stategy of lemma 13 in \cite{nedich2023huber} to arrive
    at our desired result.
    First, we will prove the inequality for $\E(f(x_k) - f(x^\star))$. Dropping the
    $-\E(\dist(x_k,\mathcal{X}))$ term from \eqref{eq:lem:upper-bound-on-f(xk)-f(xstar)-and-dist(xk)},
    and multiplying both sides by $\eta_k^{-2}$, we get
    \begin{align*}
        2\eta_k^{-1} \E(f(x_{k}) - f(x^\star))
        &\leq
        \eta_k^{-2}(1-\mu\eta_k)\E\norm{x^\star - x_k}^2 - \eta_k^{-2}\E \norm{x^\star - x_{k+1}}^2 \\
            &\phantom{===}+ M_k^2 + 2\eta_k^{-1}\gamma_k\alpha_k.
    \end{align*}
    For the choice of step size $\eta_k = 2/(\mu k)$, it holds that
    \[
        \frac{1 - \mu\eta_k}{\eta_k^2} = \frac{\mu^2 k^2 (1 - 2/k)}{4}
        = \frac{\mu^2 (k^2 - 2k)}{4} = \frac{\mu^2 ((k - 1)^2 - 1)}{4}
        \leq \frac{\mu^2 (k - 1)^2}{4}
        = \eta_{k-1}^{-2}.
    \]
    Setting $e_j := \eta_{j-1}^{-2} \E\norm{x^\star - x_j}^2$ for all $j\in\N$, we thus have
    \[
        2\eta_k^{-1} \E(f(x_{k}) - f(x^\star))
        \leq
        e_k - e_{k+1} + M_k^2 + 2\eta_k^{-1}\gamma_k\alpha_k.  
    \]
    Summing both sides over $k = k_0,\dots,K$ for $K\in\N$ yields
    \begin{align*}
        2\sum_{k=k_0}^{K} \eta_k^{-1} \E(f(x_{k}) - f(x^\star))
        &\leq
        e_{k_0} - e_{K} + \sum_{k=k_0}^{K} M_k^2 + 2\sum_{k=k_0}^{K}\eta_k^{-1}\gamma_k\alpha_k \\
        &\leq
        e_{k_0} + \sum_{k=k_0}^{K} M_k^2 + 2\sum_{k=k_0}^{K}\eta_k^{-1}\gamma_k\alpha_k,
    \end{align*}
    where we used $e_K \geq 0$ in the second step.
    We define $S_{t, k} := \sum_{i=t}^{k} \eta_i^{-1}$, $S_k := S_{1, k}$,
    and $\bar{x}_{t, k} := S_{t, k}^{-1} \sum_{i=t}^{k} \eta_i^{-1} x_i$ for $t,k\in\N$.
    Using convexity of $f$, we get
    \begin{align*}
        \E(f(\bar{x}_{k_0,K}) - f(x^\star))
        &\leq
        S_K^{-1}\sum_{k=k_0}^{K} \eta_k^{-1} \E(f(x_{k}) - f(x^\star)) \\
        &\leq
        \frac{e_{k_0}}{2S_{k_0, K}} + \frac{\sum_{k=k_0}^{K} M_k^2}{2S_{k_0, K}} + \frac{\sum_{k=k_0}^{K}\eta_k^{-1}\gamma_k\alpha_k}{S_{k_0, K}},
    \end{align*}
    for all $K\in\N$, as desired.
    Note that 
    % \[
    %     S_K = S_{0, k_1} + S_{k_1, K}
    %     \enspace\Longrightarrow\enspace
    %     1 = \frac{S_{0, k_1}}{S_K} + \frac{S_{k_1, K}}{S_K},
    % \]
    % thus
    \[
        \bar{x}_K = \frac{S_{1, k_0-1}}{S_K} \bar{x}_{1, k_0-1} + \frac{S_{k_0, K}}{S_K} \bar{x}_{k_0, K}
    \]
    and
    \[
        \frac{S_{1, k_0-1}}{S_K} + \frac{S_{k_0, K}}{S_K} = 1,
    \]
    hence, using convexity of $f$ again, we have
    \[
        \E(f(\bar{x}_K) - f(x^\star)) \leq
        \frac{S_{1, k_0-1}}{S_K} \E(f(\bar{x}_{1, k_0-1}) - f(x^\star))
        + \frac{S_{k_0, K}}{S_K} \E(f(\bar{x}_{k_0, K}) - f(x^\star)).
    \]
    Combining with the latest bound on $\E(f(\bar{x}_{k_0, K}) - f(x^\star))$ yields
    \[
        \E(f(\bar{x}_K) - f(x^\star))
        \leq
        \frac{S_{1, k_0-1}}{S_K} \E(f(\bar{x}_{1, k_0-1}) - f(x^\star))
        +
        \frac{e_{k_0}}{2S_{K}} + \frac{\sum_{k=k_0}^{K} M_k^2}{2S_{K}} + \frac{\sum_{k=k_0}^{K}\eta_k^{-1}\gamma_k\alpha_k}{S_{K}}.
    \]
    Next, we will derive the desired bound for $\dist(x_k, \mathcal{X})$.
    Again fix $k \geq k_0$.
    For the choice $x := \proj(x_k)$, inequality \eqref{eq:lem:upper-bound-on-f(xk)-f(xstar)-and-dist(xk)} gives us
    \begin{align*}
        2\eta_k\E(f(x_{k}) - f(\proj(x_k)))
        &\leq
        (1-\mu\eta_k)\E(\dist(x_k, \mathcal{X})^2) - \E\norm{\proj(x_k) - x_{k+1}}^2 \\
            &\phantom{===}+ \eta_k^2 M_k^2 + 2\eta_k\gamma_k\alpha_k
            - 2\tau \eta_k \gamma_k \E(\dist(x_k, \mathcal{X})).
    \end{align*}
    Note that $\E\norm{\proj(x_k) - x_{k+1}}^2 \geq \E(\dist(x_{k+1}, \mathcal{X})^2)$,
    hence rearranging the above yields
    \begin{align*}
        \E(\dist(x_{k+1}, \mathcal{X})^2) &\leq
        (1-\mu\eta_k)\E(\dist(x_k, \mathcal{X})^2) + \eta_k^2 M_k^2 + 2\eta_k\gamma_k\alpha_k \\
            &\phantom{===} - 2\tau \eta_k \gamma_k \E(\dist(x_k, \mathcal{X})) + 2\eta_k\E(f(\proj(x_k)) - f(x_{k})).
    \end{align*}
    Using almost sure boundedness of $(x_k)_{k\in\N}$,
    local Lipschitz continuity of $f$ (\cref{lem:basic-properties}), and Cauchy-Schwarz,
    there exists a constant $M\in[0,\infty)$,
    such that
    \[
        f(\proj(x_k)) - f(x_{k}) \leq M \dist(x_k, \mathcal{X}) \enspace (\text{a.\,s.}).
    \]
    Combining with the previous inequality and gathering terms involving $\dist(x_k,\mathcal{X})$,
    we arrive at
    \begin{align*}
        \E(\dist(x_{k+1}, \mathcal{X})^2) &\leq
        (1-\mu\eta_k)\E(\dist(x_k, \mathcal{X})^2) + \eta_k^2 M_k^2 + 2\eta_k\gamma_k\alpha_k \\
            &\phantom{===}
            - 2\eta_k (\tau \gamma_k - M) \E(\dist(x_k, \mathcal{X})).
    \end{align*}
    Since $\gamma_k \uparrow \infty$, there exists $k_1\in\N$ such that
    $\tau_k\gamma_k - M \geq \tau\gamma_k/2$ for all natural numbers $k\geq k_1$. Thus,
    \begin{align*}
        \E(\dist(x_{k+1}, \mathcal{X})^2) &\leq
        (1-\mu\eta_k)\E(\dist(x_k, \mathcal{X})^2) + \eta_k^2 M_k^2 + 2\eta_k\gamma_k\alpha_k \\
            &\phantom{===}
            - \tau \eta_k\gamma_k \E(\dist(x_k, \mathcal{X})),
    \end{align*}
    for all $k \geq k_1$.
    Multiplying both sides by $\gamma_k^{-1}\eta_k^{-2}$ yields
    \begin{align*}
        \gamma_k^{-1}\eta_k^{-2} \E(\dist(x_{k+1}, \mathcal{X})^2)
        &\leq
        \gamma_k^{-1}\eta_k^{-2}(1-\mu\eta_k)\E(\dist(x_k, \mathcal{X})^2) + \gamma_k^{-1} M_k^2 \\
            &\phantom{===}
            + 2\eta_k^{-1}\alpha_k - \tau \eta_k^{-1} \E(\dist(x_k, \mathcal{X})),
    \end{align*}
    for all $k \geq k_1$.
    We have already shown that $\eta_k^{-2} (1 - \mu\eta_k) \leq \eta_{k-1}^2$ for all $k\in\N$.
    Also, since $\gamma_k$ is nondecreasing, we have $\gamma_k^{-1} \geq \gamma_{k+1}^{-1}$ for all $k\in\N$.
    Combining these two facts, we get
    \begin{align*}
        \gamma_{k+1}^{-1}\eta_k^{-2} \E(\dist(x_{k+1}, \mathcal{X})^2)
        &\leq
        \gamma_k^{-1}\eta_{k-1}^{-2}\E(\dist(x_k, \mathcal{X})^2) + \gamma_k^{-1} M_k^2 \\
            &\phantom{===}
            + 2\eta_k^{-1}\alpha_k - \tau \eta_k^{-1} \E(\dist(x_k, \mathcal{X})),
    \end{align*}
    for all $k\geq k_1$.
    Setting $d_j := \gamma_j^{-1} \eta_{j-1}^{-2} \E(\dist(x_j, \mathcal{X})^2) \, \forall j\in\N$ and rearranging again, we obtain
    \[
        \tau \eta_k^{-1} \E(\dist(x_k, \mathcal{X}))
        \leq
        d_k - d_{k+1} + \gamma_k^{-1}M_k^2 + 2\eta_k^{-1}\alpha_k,
    \]
    for all $k\geq k_1$. Summing over $k=k_1,\dots, K$ for $K\in\N$, we have
    \begin{align*}
        \tau \sum_{k=k_1}^{K} \eta_k^{-1} \E(\dist(x_k, \mathcal{X}))
        &\leq
        \sum_{k=k_1}^{K} (d_k - d_{k+1}) + \sum_{k=k_1}^{K} \gamma_k^{-1}M_k^2 + 2 \sum_{k=k_1}^{K}\eta_k^{-1}\alpha_k \\
        &=
        d_0 - d_K + \sum_{k=k_1}^{K} \gamma_k^{-1}M_k^2 + 2 \sum_{k=k_1}^{K}\eta_k^{-1}\alpha_k \\
        &\leq
        d_0 + \sum_{k=k_1}^{K} \gamma_k^{-1}M_k^2 + 2 \sum_{k=k_1}^{K}\eta_k^{-1}\alpha_k,
    \end{align*}
    where we used $d_K \geq 0$ in the last step.
    The distance functional $x\mapsto\dist(x,\mathcal{X})$
    is convex (\textcolor{red}{TODO: Show this in an example.}), so if we divide both sides of
    the above inequality by $\tau \cdot S_{k_1, K}$, we obtain
    \[
        \E(\dist(\bar{x}_{k_1, K}))
        \leq
        \frac{\tau^{-1}d_{k_1}}{S_{k_1,K}}
            + \frac{\tau^{-1}\sum_{k=k_1}^{K } \gamma_k^{-1}M_k^2}{S_{k_1,K}}
            + \frac{2\tau^{-1}\sum_{k=k_1}^{K}\eta_k^{-1}\alpha_k}{S_{k_1,K}}.
    \]
    To derive a bound for $\E(\dist(\bar{x}_K))$, we proceed similarly as before. We have
    % \[
    %     S_K = S_{0, k_1} + S_{k_1, K}
    %     \enspace\Longrightarrow\enspace
    %     1 = \frac{S_{0, k_1}}{S_K} + \frac{S_{k_1, K}}{S_K},
    % \]
    % thus
    \[
        \bar{x}_K = \frac{S_{1, k_1-1}}{S_K} \bar{x}_{1, k_1-1} + \frac{S_{k_1, K}}{S_K} \bar{x}_{k_1, K}
    \]
    and
    \[
        \frac{S_{1, k_1-1}}{S_K} + \frac{S_{k_1, K}}{S_K} = 1.
    \]
    Combining with the latest bound on $\E(\dist(\bar{x}_{k_1, K}))$, we obtain the desired bound
    \begin{align*}
        \E(\dist(\bar{x}_{K}))
        &\leq
        \frac{S_{1, k_1-1}}{S_K} \E(\dist(\bar{x}_{1, k_1-1}))
            + \frac{S_{k_1, K}}{S_K} \E(\dist(\bar{x}_{k_1, K})) \\
        &\leq
        \frac{S_{1, k_1-1}}{S_K} \E(\dist(\bar{x}_{1, k_1-1}))
            + \frac{\tau^{-1}d_{k_1}}{S_{K}}
            + \frac{\tau^{-1}\sum_{k=k_1}^{K} \gamma_k^{-1}M_k^2}{S_{K}}
            + \frac{2\tau^{-1}\sum_{k=k_1}^{K}\eta_k^{-1}\alpha_k}{S_{K}}
    \end{align*}
    for all $K \geq k_1$.
    This concludes the proof.
\end{proof}

\begin{theorem}[Convergence rates in expectation]
    \label{thm:convergence-rates-in-expectation}
    Let \cref{ass:basic-assumptions,ass:compact-support-without-zero,ass:slater-point} hold
    and define $\eta_k := 2/(\mu k)$ for all $k\in\N$, $\eta_0 := 2/\mu$.
    Then, for any $e\in(0,\infty)$, there exist sequences $(\gamma_k)_{k\in\N_0}$,
    $(h_k)_{k\in\N_0}$, $(\beta_k)_{k\in\N_0}$
    such that the iterates $(\bar{x}_k)_{k\in\N}$
    generated from \cref{alg:seqprox-sgd} with parameters $(\eta_k)_{k\in\N_{0}}$,
    $(\gamma_k)_{k\in\N_0}$, $(h_k)_{k\in\N_0}$, $(\beta_k)_{k\in\N_0}$ satisfy
    \[
        \E| f(\bar{x}_{K}) - f(x^\star) | = \mathcal{O} \Big( \frac{\log^{2e}(K)}{K} \Big).
    \]
    Further, it also holds that
    \[
        \E(\dist(\bar{x}_{K}, \mathcal{X})) = \mathcal{O} \Big( \frac{\log^{e}(K)}{K} \Big).
    \]
\end{theorem}
\begin{proof}
    \Cref{lem:upper-bound-on-f(xk)-f(xstar)-and-dist(xk)} gives us an upper bound for $\E(f(\bar{x}_k) - f(x^\star))$.
    By convexity, we have for all $k\in\N$
    \begin{align*}
        f(\bar{x}_k) - f(x^\star) &\geq \langle \tilde{\nabla} f(x^\star), \bar{x}_k - x^\star \rangle \\
                            &= \langle \tilde{\nabla} f(x^\star), \bar{x}_k - \proj(\bar{x}_k) \rangle
                                + \langle \tilde{\nabla} f(x^\star), \proj(\bar{x}_k) - x^\star \rangle,
    \end{align*}
    for any subgradient $\tilde{\nabla} f(x^\star) \in \partial f(x^\star)$. Optimimality of $x^\star$
    for $f$ on $\mathcal{X}$ implies that there exists a subgradient
    $\tilde{\nabla} f(x^\star) \in \partial f(x^\star)$ such that
    $\langle \tilde{\nabla} f(x^\star), \proj(\bar{x}_k) - x^\star \rangle \geq 0$.
    Combining these facts
    with the above inequality, and applying Cauchy-Schwarz,
    we obtain
    \begin{equation}
        \label{eq:proof:thm:convergence-rates-in-expectation}
        f(\bar{x}_k) - f(x^\star) \geq - \norm{\tilde{\nabla} f(x^\star)} \dist(\bar{x}_k, \mathcal{X}),
    \end{equation}
    for all $k\in\N$. From \cref{lem:upper-bound-on-f(xk)-f(xstar)-and-dist(xk)}
    we know that there exists $k_1\in\N$ such that for all $K\in\N$ with $K\geq k_1$,
    it holds that
    \[
        \E(\dist(\bar{x}_K, \mathcal{X}))
        \leq
        \frac{S_{1, k_1}}{S_K} \E(\dist(\bar{x}_{0, k_1}))
            + \frac{\tau^{-1}d_{k_1}}{S_{K}}
            + \frac{\tau^{-1}\sum_{k=k_1}^{K - 1} \gamma_k^{-1}M_k^2}{S_{K}}
            + \frac{2\tau^{-1}\sum_{k=k_1}^{K - 1}\eta_k^{-1}\alpha_k}{S_{K}},
    \]
    where $S_K := \sum_{k=0}^{K-1} \eta_k^{-1}$ and $M_k^2 = \mathcal{O}(\beta_k^{-1}(1 + \gamma_k^2) + \gamma_k^2)$.
    Since $\eta_k^{-1} = (\mu k)/2$, it follows that $S_K = \mathcal{O}(K^2)$. Also, since
    $\alpha_k = \mathcal{O}(k)$, we have $\eta_k^{-1} \alpha_k \leq c$ for some $c\in (0,\infty)$.
    Together with $\gamma_k^{-1}M_k^2 = \mathcal{O}(\gamma_k)$, we therefore have
    \begin{align*}
        \E(\dist(\bar{x}_K, \mathcal{X}))
        &=
        \mathcal{O}(K^{-2})
            + \mathcal{O}\Big( K^{-2} \sum_{k=1}^{K - 1} \log^e(k) \Big)
            + \mathcal{O}( K^{-1} ) \\
        &=
        \mathcal{O}(K^{-2})
            + \mathcal{O}( \log^e(K) / K )
            + \mathcal{O}( K^{-1} )  \\
        &= \mathcal{O} \Big( \frac{\log^{e}(K)}{K} \Big).
    \end{align*}
    Combining with \eqref{eq:proof:thm:convergence-rates-in-expectation}, we obtain
    \[
        \E( f(\bar{x}_k) - f(x^\star) )
        \geq - \norm{\nabla f(x^\star)} \cdot \mathcal{O} \Big( \frac{\log^{2e}(K)}{K} \Big).
    \]
    The upper bound from \cref{lem:upper-bound-on-f(xk)-f(xstar)-and-dist(xk)} reads
    \[
        \E(f(\bar{x}_{K}) - f(x^\star))
        \leq
        \frac{S_{1, k_0}}{S_K} \E(f(\bar{x}_{0, k_0}) - f(x^\star)) +
        \frac{e_{k_0}}{2S_K} + \frac{\sum_{k=k_0}^{K-1} M_k^2}{2S_K} + \frac{\sum_{k=k_0}^{K-1}\eta_k^{-1}\gamma_k\alpha_k}{S_K},
    \]
    and since $M_k^2 = \mathcal{O}(\log^{2e} k)$ and $\eta_k^{-1}\gamma_k\alpha_k = \mathcal{O}(\log^{e} k)$,
    it follows that
    \begin{align*}
        \E(f(\bar{x}_{K}) - f(x^\star))
        &=
        \mathcal{O}(K^{-2})
        + \mathcal{O} \Big( K^{-2} \sum_{k=1}^{K-1} \log^{2e}(k) \Big)
        + \mathcal{O} \Big( K^{-2} \sum_{k=1}^{K-1} \log^{e}(k) \Big) \\
        &=
        \mathcal{O} \Big( \frac{\log^{2e}(K)}{K} \Big).
    \end{align*}
    Putting the two bounds together, we obtain
    \[
        \E| f(\bar{x}_{K}) - f(x^\star) | = \mathcal{O} \Big( \frac{\log^{2e}(K)}{K} \Big),
    \]
    as desired.
\end{proof}

\section{High-probability guarantees}

In the previous sections we proved $\mathcal{O}(\log k / k)$ convergence to zero of the expected
suboptimality, $\E|f(\bar{x}_k) - f(x^\star)|$, and the expected subfeasibility, $\dist(\bar{x}_k, \mathcal{X})$,
of the sequence $(\bar{x}_{k})_{k\in\N}$ generated by \cref{alg:seqprox-sgd}, under
suitable choice of parameters $(\eta_k)_{k\in\N_0}$, $(\gamma_k)_{k\in\N_0}$, $(\beta_k)_{k\in\N_0}$,
and $(h_k)_{k\in\N_0}$.
We will now turn our attention to establishing \textit{high-probability guarantees}, in contrast
to guarantees that hold only in expectation. In particular, we will investigate under which
conditions we can guarantee that an iterate $\bar{x}_k$ is both a) close to the feasible set
and b) has function value close to the optimal value $f(x^\star)$ with high probability.
This notion of "closeness with high probability" is formalized in the following definition.
\begin{definition}[$(\epsilon, \delta)$-solution]
    Let $\epsilon\in(0,\infty)$ and $\delta\in(0, 1)$.
    We call a random variable $x\colon\Omega\to\R^n$ an \textbf{$(\epsilon, \delta)$-solution} of \eqref{eq:main-problem-copy},
    if
    \[
        \Prob\Big( \max\big( | f(x) - f(x^\star) |, \dist(x,\mathcal{X}) \big) \geq \epsilon \Big) \leq \delta.
    \]
\end{definition}
In other words, an iterate $x_k$ of \cref{alg:seqprox-sgd} is an $(\epsilon, \delta)$-solution, if we can guarantee that
$| f(x_k) - f(x^\star) | < \epsilon$ and $\dist(x_k, \mathcal{X}) < \epsilon$ with probability
greater than $1 - \delta$.
% One thing to note is that the decision to consider $\pi(x)$
% instead of $\dist(x, \mathcal{X})$ is purposeful, because
% this allows $(\epsilon, \delta)$-solutions to exist even
% if $\mathcal{X} = \emptyset$. On the other hand,
% if $\mathcal{X} \neq \emptyset$ and $X$ is an $(\epsilon, \delta)$-solution,
% it follows from \cref{lem:lower-bound-on-penalties-by-distance-to-feasibility}
% that $\dist(X,\mathcal{X}) < c \cdot \epsilon$ with probability greater than $1 - \delta$
% for some constant $c\in(0,\infty)$ independent of $\epsilon$.
We can now state the central questions we aim to answer in this section as follows:
\begin{enumerate}
    \item For what choice of parameters can we guarantee that the sequence $(\bar{x}_k)_{k\in\N}$ generated by \cref{alg:seqprox-sgd}
        reaches an $(\epsilon, \delta)$-solution of \eqref{eq:main-problem-copy} as quickly as possible,
        for any given choice of $\epsilon$ and $\delta$?
    \item What is the relatioship between $\epsilon,\delta$, and the number of iterations of \cref{alg:seqprox-sgd}
        needed to reach an $(\epsilon, \delta)$-solution of \eqref{eq:main-problem-copy}?
\end{enumerate}
To resolve these questions, we will first need a more general version of \cref{lem:one-step-improvement}, which
is established in the next lemma.
We will then proceed similarly to how we did in the previous section,
though we will need to carry around an extra term throughout the calculations.
\begin{lemma}[One-step improvement II]
    Let $\rho\in(0,1)$ and $\eta_k\in(0, \rho L^{-1}]$ for all $k\in\N_0$. Then
    the iterates $(x_k)_{k\in\N}$ generated by \cref{alg:seqprox-sgd} with step size
    schedule $(\eta_k)_{k\in\N_0}$ satisfy
    \begin{align*}
        2\eta_k(f_k(x_{k+1}) - f_k(x))
        &\leq ( 1 - \mu\eta_k ) \norm{x - x_k}^2 + 2\eta_k\langle z_k, x_k - x \rangle 
            - \norm{x - x_{k+1}}^2 \\
        &\phantom{===} +  \frac{8 G^2\eta_k^2\gamma_k^2}{1 - \rho} + \frac{2\eta_k^2}{1 - \rho}\norm{z_k}^2
    \end{align*}
    for all $k\in\N$.
\end{lemma}
\begin{proof}
    This is just inequality \eqref{eq:proof:lem:one-step-improvement-2}. Since we
    make the same assumptions in this lemma as in \cref{lem:one-step-improvement},
    we can simply follow the exact same steps until we arrive at the desired result.
\end{proof}
\begin{lemma}
    Let \cref{ass:step-sizes-and-penalties} hold.
    For all $k\in\N$, it holds that
    \[
        \norm{z_k}^2 \leq 8L^2 + 8\gamma_k^2G^2,
    \]
    with probability one, where $L, G\in(0,\infty)$.
\end{lemma}
\begin{proof}
    Using the inequality $(a + b + c + d)^2 \leq 4 (a^2 + b^2 + c^2 + d^2)$ $\forall a,b,c,d\in\R$, we have
    \begin{align*}
        \norm{z_k}^2
        &= \norm{g_k - \nabla \psi_k(x_k)}^2 \\
        &= \norm{\nabla F_{\xi_k}(x_k) + \gamma_k h_k(x_k; A(\xi), b(\xi))
            + \E(\nabla F_{\xi_k}(x_k)) + \gamma_k \nabla \pi_k(x_k)}^2 \\
        &\leq 4 \big( \norm{\nabla F_{\xi_k}(x_k)}^2 + \gamma_k^2 \norm{\nabla h_k(x_k; A(\xi), b(\xi))}^2 \\
            &\phantom{===}+ \norm{\E( \nabla F_{\xi_k}(x_k) )}^2 + \gamma_k^2 \norm{\nabla \pi_k(x_k)}^2 \big),
    \end{align*}
    where we also used $\nabla \E(F_{\xi_k}(x_k)) = \E(\nabla F_{\xi_k}(x_k))$. This follows from
    almost-sure smoothness of $F_\xi$, which we can also use,
    along with almost-sure boundedness of $(x_k)_{k\in\N}$, to locally bound
    \[
        \norm{\nabla F_{\xi_k}(x_k)} \leq L_{\text{loc}}
        \enspace
        \text{and}
        \enspace
        \norm{\E(\nabla F_{\xi_k}(x_k))} \leq \E\norm{ \nabla F_{\xi_k}(x_k) } \leq L_{\text{loc}}.
    \]
    Therefore
    \[
        \norm{\nabla F_{\xi_k}(x_k) + \gamma_k \nabla h_k(x_k; A(\xi), b(\xi))}
        \leq
        L_{\text{loc}} + \gamma_k G
    \]
    and, combining with
    \[
        \sup_{k\in\N}\sup_{x\in\R^n} \norm{\nabla h_k(x; A(\xi), b(\xi))} \leq G,
        \enspace
        \text{and}
        \enspace
        \sup_{k\in\N}\sup_{x\in\R^n} \norm{\nabla \pi_k(x)} \leq G
    \]
    for a constant $G\in(0,\infty)$ (\cref{lem:basic-properties}), we obtain
    \[
        \norm{z_k}^2
        \leq
        4(2 L_{\text{loc}}^2 + 2\gamma_k^2G^2) = 8L_{\text{loc}}^2 + 8\gamma_k^2G^2,
    \]
    as desired.
\end{proof}

\section{Infeasible problems}

As we have seen in the SVM example (\textcolor{red}{TODO}), some problems
of interest may not be feasible. Yet, our methods can still be applied in those cases.
The question is then: What do the iterates converge to, if anything?
\begin{definition}
    Let $\delta \in [0,1]$. A point $x\in\R^n$ is called \textbf{$\delta$-feasible}, if
    \[
        \Prob(A(\xi)x - b(\xi) > 0) \leq \delta. 
    \]
    The \textbf{$\delta$-set}, denoted $\mathcal{X}_\delta$, is the set of all $\delta$-feasible points.
    A point $x\in\R^n$ is called \textbf{maximally feasible}, if
    there exists $\delta\in[0,1]$ such that $(x, \delta)$ solves
    \[
        \min_{(x, \delta)\in\R^n\times[0,1]} \delta \quad \text{s.\,t.} \enspace x\in \mathcal{X}_\delta.
    \]
\end{definition}
\noindent
\textbf{Conjecture:} Consider the following two statements:
\begin{enumerate}
    \item For any $\delta\in(0,1]$, the sequence of iterates $(x_k)_{k\in\N}$
    is eventually contained in $\mathcal{X}_\delta$ in probability.
    \item The sequence of iterates $(x_k)_{k\in\N}$ converges to a maximally feasible point in probability.
\end{enumerate}
At least one of these two statements must hold, regardless of whether \eqref{eq:main-problem-copy}
is feasible or not. Both statements hold iff. problem \eqref{eq:main-problem-copy} is feasible.
%\section{Exponential moving averages}


% \section{A general consistency result}

% In this section, we want to analyze general conditions that
% ensure $\lim_{k\to\infty} x_k^\star = x^\star$, which we refer
% to as consistency of the sequence $(x_k^\star)_{k\in\N}$.
% This will help us in two ways: First, it gives a general
% guide on what constitutes reasonable sequences $(\pi_k)_{k\in\N}$.
% Second, in subsequent analysis, it will be highly useful
% to have condtions that ensure consistency, because this
% will allow us to use boundedness of the sequence $(x_k^\star)_{k\in\N}$
% in our analysis.
% The precise result that we will prove in this section is the following.
% \begin{theorem}
%     In problem \eqref{eq:main-problem}, assume that the following hold:
%     \begin{enumerate}
%         \item There exists $\mu\in(0,\infty)$ such that
%             $f$ is $\mu$-strongly convex;
%         \item The sequence $(\gamma_k)_{k\in\N}$
%             is positive and unbounded;
%         \item There exists
%             at least one feasible point;
%         \item $(\pi_k)_{k\in\N}$ is a sequence of
%             nonnegative convex functions,
%             such that $\gamma_{k+1}\,\pi^{k+1}(x) \geq \gamma_{k}\,\pi^{k}(x)$
%             for all $x\in\R^d$ and $k\in\N$ (\textcolor{red}{TODO: NOT NEEDED ACTUALLY, assume
%             instead existence of a sequence $c_k$ such that $\pi_k(x) \leq c_k$ for all feasible $x$,
%             and $\gamma_k c_k \to 0$}),
%             and $\lim_{k\to\infty} \gamma_k \pi_k(x) = 0$ for all feasible $x\in\R^d$. Moreover, we assume the existence
%             of a continuous function $\pi^\infty\colon\R^d\to [0, \infty)$
%             such that $\pi^\infty \leq \pi_k$ for all $k\in\N$
%             and $\pi^\infty(x) = 0$ if and only if $x$ is feasible for \eqref{eq:new-model-problem}.
%     \end{enumerate}
%     Then, we have $\lim_{k\to\infty} x^\star_k = x^\star$ and
%     $\lim_{k\to\infty} f^k(x^\star_k) = f(x^\star)$.
% \end{theorem}

% \begin{lemma}
%     \label{lem:radially-unbounded-proper-cont-then-attains-minimum}
%     Let $h\colon \R^d \to \R$ be a coercive and continuous function, and let $X \subset \R^d$ be nonempty
%     and closed. Then $h$ attains a minimum over $X$, i.\,e.
%     there exists $x^\star\in X$ such that $h(x^\star) = \inf_{x\in X} h(x)$.
% \end{lemma}
% \begin{proof}
%     Let $x_0\in X$. Since $h$ is coercive, there exists $r > 0$
%     such that $h(x) \geq h(x_0)$ for all $x\in\R^d$ with $|\!|x|\!| > r$,
%     therefore any minimum of $h$ -- if it exists -- must be contained in the closed ball of
%     radius $r$ around $0$, which we denote by
%     $B_r$. In particular, for $C := X \cap B_r$ we have
%     \[
%         \inf_{x\in X} h(x) = \inf_{x\in C} h(x).
%     \]
%     By continuity of $h$, its domain must be a closed set, which implies that $C$ is compact.
%     Assume now that $h$ does not attain a minimum on $C$.
%     Then there must exist a sequence $(x_k)_{k\in\N} \subset C$
%     such that $\lim_{k\to\infty} h(x_k) = \inf_{x\in C} h(x)$.
%     Continuous functions map compact sets to compact sets,
%     hence $\inf_{x\in C} h(x) \in h(C)$ and thus there must exist some
%     $x^\star\in C$ such that $h(x^\star) = \inf_{x\in C} h(x)$.
% \end{proof}
% \begin{lemma}
%     \label{lem:strongly-convex-implies-coercive}
%     Let $h\colon\R^d\to\R$ be strongly convex and differentiable. Then $h$ is coercive.
% \end{lemma}
% \begin{proof}
%     Let $x^\star$ be the minimizer of $h$ (\cref{prop:strongly-convex-implies-unique-minimizer}).
%     By \cref{prop:strongly-convex-subdifferentiable-bound}, we have
%     \[
%         h(x) \geq h(x^\star) + \frac{\mu}{2}|\!|x-x^\star|\!|^2
%     \]
%     for all $x\in\R^d$ and some $\mu\in (0, \infty)$. Letting $|\!|x|\!|\to\infty$, we see that $h(x) \to \infty$.
% \end{proof}
% \begin{lemma}
%     \label{lem:radially-unbounded-f-and-f(x^n)-bounded-implies-x^n-bounded}
%     Let $h\colon \R^d \to \R$ be a coercive function
%     If $\{\, h(u_k) \mid k\in\N \,\}$
%     is bounded for some sequence
%     $(u_k)_{k\in\N} \subset \R^d$, then $(u_k)_{k\in\N}$ must also be bounded.
% \end{lemma}
% \begin{proof}
%     Assume $(u_k)_{k\in\N}$ is not bounded. Then there must exist some subsequence $(u_{k_r})_{r\in\N}$
%     such that $|\!|u_{k_r}|\!| \to \infty$ for $r\to\infty$. By coercivity, this would
%     imply that $\lim_{k\to\infty} h(u_k) = \infty$, contradicting our assumption
%     that $\{\, h(u_k) \mid k\in\N \,\}$ is bounded. Hence, $(u_{k})_{k\in\N}$ must be bounded.
% \end{proof}
% \begin{lemma}
%     \label{lem:if-every-subsequence-contains-convergent-subsequence-then-sequence-converges}
%     Let $U := \{\, u_k \mid k\in\N \,\}$ be a subset of $\R^d$.
%     Suppose that any subsequence of $U$ contains a subsequence
%     that converges to $u\in\R^d$.
%     Then $u_k \to u$ for $k\to\infty$.
% \end{lemma}
% \begin{proof}
%     Assume that $u_k \not\to u$.
%     Then there must exist some $\epsilon > 0$ and a sequence of natural numbers $k_1 < k_2 < \dots$ such that
%     \begin{equation}
%         \label{proof:lem:if-every-subsequence-contains-convergent-subsequence-then-sequence-converges:ineq}
%         |\!|u_{k_r} - u|\!| \geq \epsilon
%     \end{equation}
%     for all $r\in\N$. However, as a subsequence of $U$, the sequence
%     $(u_{k_r})_{r\in\N}$
%     must simultanously contain a subsequence that converges to $u$, which contradicts
%     \eqref{proof:lem:if-every-subsequence-contains-convergent-subsequence-then-sequence-converges:ineq}.
%     Thus, our assumption $u_k \not\to u$ must be false.
% \end{proof}